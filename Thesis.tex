%%%%%%%%%%%%%%%%%%%%%%%%%%%%%%%%%%%%%%%%%%%%%%%%%%%%%%%%%%%%%%%%%%%%%%%%%%%%%%%%
% Archivo: Thesis.tex
% Creado: 11-09-2017 - 08:15 por Narciso López
% Modificado:
% 16-03-2018 - 13:11 por Narciso López
%
% Archivo LaTeX destinado a servir como plantilla para realizar la memoria de 
% Ingeniería Civil Informática o la Tesis de Magíster/Doctorado en el
% Departamento de Ingeniería Informática y Ciencias de la Computación de
% la Universidad de Concepción
% Para más información sobre las reglas nuevas para escribir una memoria o 
% tesis visitar:
% http://www.bibliotecas.udec.cl/sites/default/files/PAUTAS_DE_NORMALIZACION_EN_LA_PRESENTACION_DE_UNA_TESIS_DE_GRADO_O_TITULACION_Mayo.pdf 

%%%%%%%%%%%%%%%%%%%%%%%%%%%%%%%%%%%%%%%%%%%%%%%%%%%%%%%%%%%%%%%%%%%%%%%%%%%%%%%%
\documentclass[12pt]{ThesisDIICC}
\usepackage{anysize} 
%\marginsize{4cm}{2.5cm}{4cm}{2.5cm} 
%\renewcommand{\rmdefault}{phv} % Arial
%\renewcommand{\sfdefault}{phv} % Arial
\usepackage[utf8]{inputenc}
\usepackage[spanish, es-tabla]{babel}

%\usepackage[toc,page]{appendix}
%\addto\captionsspanish{%
%  \renewcommand\appendixname{Anexo}
%  \renewcommand\appendixpagename{Anexos}
%  \renewcommand\appendixtocname{Anexos}
%}


%%%%%%%%%%%%%%%%%%%%%%%%%%%%%%%%%%%%%%%%%%%%%%%%%%%%%%%%%%%%%%%%%%%%%%%%%%%%%%%%
%
% Paso 1: Agrega tus paquetes aquí
%
% http://math.kangwon.ac.kr/~yhpark/tex/packages.html
%%%%%%%%%%%%%%%%%%%%%%%%%%%%%%%%%%%%%%%%%%%%%%%%%%%%%%%%%%%%%%%%%%%%%%%%%%%%%%%%
\usepackage{setspace}

%\graphicspath{ {../images/} }

\usepackage[table, dvipsnames]{xcolor} % pretty tables
\usepackage{multirow} % multicolumn on tables

%\usepackage[dvipsnames]{xcolor}
%\definecolor{Green1}{HTML}{005a32}
%\colorlet{Mycolor1}{green!10!orange!90!}
%\definecolor{Mycolor2}{HTML}{00F9DE}

\scriptsize
\setlength\tabcolsep{3pt} % let tabular* figure out intercolumn whitespace

\usepackage{booktabs} % Table toprule middlerule bottomrule

\usepackage{amsmath} % align equations
\usepackage{amssymb} % void symbol
\usepackage{mathtools} % \ceil
\DeclarePairedDelimiter{\ceil}{\lceil}{\rceil}

\usepackage{bm} % bold math symbols

\usepackage{amsthm} % Theorems (definitions)

\usepackage{amsfonts} % more math symbols

\newtheorem{definition}{Definición}[chapter] % Definition with section numbering
\newtheorem{problem}{Problema}[chapter] % Definition with section numbering

%\usepackage{pdfpages} %include pdf

\usepackage{tikz} % Draw lines


\usepackage{algorithm} % Writing pseudocode
\usepackage{algorithmic} % Writing pseudocode
%\input{spanishAlgorithmic} % Archivo de traducción (https://rosapolis.net/2008/04/21/escribir-algoritmos-en-latex/index.html)

\numberwithin{algorithm}{chapter} % Correct numbering of algorithms

%\usepackage{caption, subcaption} % subcaption for subfigures

\usepackage[%  
    colorlinks=false,
    pdfborder={0 0 0},
    %linkcolor=black
]{hyperref} % Links on numbers

\usepackage[dvipsnames]{xcolor}

\definecolor{azul}{HTML}{AAFFFF}
\definecolor{amarillo}{HTML}{FFFFAA}
\definecolor{verde}{HTML}{E2FFAA}
\definecolor{rojo}{HTML}{E75A7C}
\definecolor{cafe}{HTML}{758ECD}
\definecolor{blanco}{HTML}{FFFFFF}

\definecolor{LLG}{gray}{0.9}
\definecolor{LG}{gray}{0.8}
\definecolor{G}{gray}{0.7}


\definecolor{color1}{HTML}{006895}
\definecolor{color2}{HTML}{00AA4F}
\definecolor{color3}{HTML}{FF3769}


%%%%%%%%%%%%%%%%%%%%%%%%%%%%%%%%%%%%%%%%%%%%%%%%%%%%%%%%%%%%%%%%%%%%%%%%%%%%%%%%
%
% Paso 2: Descomentar si es un borrador (draft)
%
%%%%%%%%%%%%%%%%%%%%%%%%%%%%%%%%%%%%%%%%%%%%%%%%%%%%%%%%%%%%%%%%%%%%%%%%%%%%%%%%
%\draft
\singlespace


%%%%%%%%%%%%%%%%%%%%%%%%%%%%%%%%%%%%%%%%%%%%%%%%%%%%%%%%%%%%%%%%%%%%%%%%%%%%%%%%
%
% Paso 3: Agrega tus definiciones aquí
%
% http://en.wikibooks.org/wiki/LaTeX/Customizing_LaTeX
%%%%%%%%%%%%%%%%%%%%%%%%%%%%%%%%%%%%%%%%%%%%%%%%%%%%%%%%%%%%%%%%%%%%%%%%%%%%%%%%
\newcommand{\ignore}[1]{}


%%%%%%%%%%%%%%%%%%%%%%%%%%%%%%%%%%%%%%%%%%%%%%%%%%%%%%%%%%%%%%%%%%%%%%%%%%%%%%%%
%
% Paso 4: Elige tu grado
%
% Escribe \eng para Ingeniería Civil Informática
% Escribe \msc para Magíster en Ciencias de la Computación
% Escribe \phd para Doctorado en Ciencias de la Computación
%%%%%%%%%%%%%%%%%%%%%%%%%%%%%%%%%%%%%%%%%%%%%%%%%%%%%%%%%%%%%%%%%%%%%%%%%%%%%%%%
\msc


%%%%%%%%%%%%%%%%%%%%%%%%%%%%%%%%%%%%%%%%%%%%%%%%%%%%%%%%%%%%%%%%%%%%%%%%%%%%%%%%
%
% Paso 5: Elige tu programa
%
% Escribe \programeng para Ingeniería Civil Informática
% Escribe \programmsc para Programa de Magíster en Ciencias de la Computación
% Escribe \programphd para Programa de Doctorado en Ciencias de la Computación
%%%%%%%%%%%%%%%%%%%%%%%%%%%%%%%%%%%%%%%%%%%%%%%%%%%%%%%%%%%%%%%%%%%%%%%%%%%%%%%%
\programmsc


%%%%%%%%%%%%%%%%%%%%%%%%%%%%%%%%%%%%%%%%%%%%%%%%%%%%%%%%%%%%%%%%%%%%%%%%%%%%%%%%
%
% Paso 6: Título de Memoria y autor
%
%%%%%%%%%%%%%%%%%%%%%%%%%%%%%%%%%%%%%%%%%%%%%%%%%%%%%%%%%%%%%%%%%%%%%%%%%%%%%%%%
\title{\bf Diseño e implementación de método de compresión de grafos basado en clustering de cliques maximales}
\author{Felipe Alberto Glaría Grego} 


%%%%%%%%%%%%%%%%%%%%%%%%%%%%%%%%%%%%%%%%%%%%%%%%%%%%%%%%%%%%%%%%%%%%%%%%%%%%%%%%
%
% Paso 7: Agrega a tu profesor/es guía/s
%
%%%%%%%%%%%%%%%%%%%%%%%%%%%%%%%%%%%%%%%%%%%%%%%%%%%%%%%%%%%%%%%%%%%%%%%%%%%%%%%%
\advisor{Lilian Salinas Ayala\\Profesor co-guía: Cecilia Hernández Rivas} 
%NOMBRE1 NOMBRE2 APELLIDO1 APELLIDO2


%%%%%%%%%%%%%%%%%%%%%%%%%%%%%%%%%%%%%%%%%%%%%%%%%%%%%%%%%%%%%%%%%%%%%%%%%%%%%%%%
%
% Paso 8: Ajusta fechas de presentación, copyright y defensa 
%
%%%%%%%%%%%%%%%%%%%%%%%%%%%%%%%%%%%%%%%%%%%%%%%%%%%%%%%%%%%%%%%%%%%%%%%%%%%%%%%%
\submitdate{Mayo, 2019} % fecha de presentación al Comité
\defensedate{Mayo, 2019}  % fecha de defensa
\copyrightyear{2019}         % archivado final


\begin{document}
\frontmatter

%%%%%%%%%%%%%%%%%%%%%%%%%%%%%%%%%%%%%%%%%%%%%%%%%%%%%%%%%%%%%%%%%%%%%%%%%%%%%%%%
%
% Paso 9: Copyright según nuevas normas de biblioteca
%
%%%%%%%%%%%%%%%%%%%%%%%%%%%%%%%%%%%%%%%%%%%%%%%%%%%%%%%%%%%%%%%%%%%%%%%%%%%%%%%%
 \begin{copyright}
 \vskip 2.5ex
 
%	El autor(es) de la tesis debe incluir una de las siguientes opciones: 
     
% a)  Ninguna  parte de  esta  tesis  puede  reproducirse  o  transmitirse  bajo  ninguna  forma  o  por ning\'un medio o procedimiento, sin permiso por escrito del autor.  

%b)  
Se  autoriza  la  reproducci\'on  total  o  parcial,  con  fines  acad\'emicos,  por  cualquier  medio  o procedimiento, incluyendo la cita bibliogr\'afica del documento. 

 \end{copyright}
 
 \clearpage


%%%%%%%%%%%%%%%%%%%%%%%%%%%%%%%%%%%%%%%%%%%%%%%%%%%%%%%%%%%%%%%%%%%%%%%%%%%%%%%%
%
% Paso 10: Página de calificaciones
%
% Descomentar para ver (opcional)
%%%%%%%%%%%%%%%%%%%%%%%%%%%%%%%%%%%%%%%%%%%%%%%%%%%%%%%%%%%%%%%%%%%%%%%%%%%%%%%% 
% \frontmatter
 
 
%%%%%%%%%%%%%%%%%%%%%%%%%%%%%%%%%%%%%%%%%%%%%%%%%%%%%%%%%%%%%%%%%%%%%%%%%%%%%%%%
%
% Paso 11: Dedicatoria
%
% Opcional
%%%%%%%%%%%%%%%%%%%%%%%%%%%%%%%%%%%%%%%%%%%%%%%%%%%%%%%%%%%%%%%%%%%%%%%%%%%%%%%%
 
 \begin{dedication}
 \vskip 2.5ex
 A mi familia...
 \end{dedication}


%%%%%%%%%%%%%%%%%%%%%%%%%%%%%%%%%%%%%%%%%%%%%%%%%%%%%%%%%%%%%%%%%%%%%%%%%%%%%%%%
%
% Paso 12: Agradecimientos
%
% Opcional
%%%%%%%%%%%%%%%%%%%%%%%%%%%%%%%%%%%%%%%%%%%%%%%%%%%%%%%%%%%%%%%%%%%%%%%%%%%%%%%%

 \begin{acknowledgements}
	Quisiera agradecer en primer lugar a mi hija Olivia, mi madre Milena, a Paulina y Claudia, mujeres elementales en mi vida durante el transcurso de este trabajo, quienes entregaron su apoyo incondicional en todo momento, y por ello les agradezco profundamente.

También agradezco a las profesoras Cecilia Hernández y Lilian Salinas. Su soporte, ayuda y comprensión en el desarrollo de esta tesis es invaluable.

De igual manera, agradezco a mis compañeros de generación del programa de Magister, con quienes compartimos y generamos una comunidad de aprendizaje fundamental para los seres sociales que somos, y así poder avanzar de manera grata y fructíera.

 \end{acknowledgements}


%%%%%%%%%%%%%%%%%%%%%%%%%%%%%%%%%%%%%%%%%%%%%%%%%%%%%%%%%%%%%%%%%%%%%%%%%%%%%%%%
%
% Paso 13: Abstract
%
% http://research.berkeley.edu/ucday/abstract.html
%%%%%%%%%%%%%%%%%%%%%%%%%%%%%%%%%%%%%%%%%%%%%%%%%%%%%%%%%%%%%%%%%%%%%%%%%%%%%%%%
\begin{abstract}
	La compresión y manejo eficiente de grandes grafos se vuelve cada día más fundamental en distintos ámbitos, desde lo macro de la representación de la Web y las redes sociales, hasta lo micro de representar componentes biológicos. Su crecimiento ha tenido un aumento constante, y con esto se eleva la exigencia en recursos para almacenarlos, junto con la complejidad de los algoritmos para procesar consultas sobre ellos.

Una alternativa a este problema es buscar una opción de representar estos grafos de manera comprimida que, además de usar menos espacio en disco, permita responder dichas consultas en el menor tiempo posible.

En este trabajo se desarrolla un método de compresión para grafos no dirigidos, que aprovecha la superposición de cliques maximales en la generación de una estructura compacta que permite manejar de manera eficiente dichos grafos, ahorrando en espacio de disco pero pagando su costo en tiempos de acceso.

Si bien el problema de obtener los cliques maximales de un grafo es muy complejo (NP-completo), existen algoritmos para grafos esparsos que logran generar el listado en poco tiempo. Además es un costo que se debe pagar una sola vez, luego de generar la estructura compacta se puede obtener nuevamente el listado de cliques directo de ella.

Finalmente se prueba el rendimiento de esta estructura propuesta con diferentes grafos, estudiando qué características deben tener para aprovechar sus ventajas, y se compara con otros algoritmos relevantes del estado del arte, comprobando su rendimiento tanto en la compresión de grafos como en resolver las consultas sobre ella.

Se identifica que para grafos altamente clusterizados la compresión logra su máxima eficiencia, ocupando menos de un bit por arco, y siempre menos que los demás algoritmos. En cuanto a tiempos de acceso, los resultados más competitivos se logran en acceso aleatorio, y se proponen líneas de investigación para mejorar esto. 

\end{abstract}


%%%%%%%%%%%%%%%%%%%%%%%%%%%%%%%%%%%%%%%%%%%%%%%%%%%%%%%%%%%%%%%%%%%%%%%%%%%%%%%%
%
% Paso 14: Introducción
%
% 
%%%%%%%%%%%%%%%%%%%%%%%%%%%%%%%%%%%%%%%%%%%%%%%%%%%%%%%%%%%%%%%%%%%%%%%%%%%%%%%%
\mainmatter


\chapter{INTRODUCCI\'ON}\label{chap:intro}
\vskip 3.0ex

%\textcolor{red}{Todavía no he cambiado nada.}

En los últimos años se ha visto un gran crecimiento en grafos de redes sociales y de la web. Por ejemplo, el número de sitios indexados por los principales motores de búsqueda en la web se estima actualmente en al menos 5,68 miles de millones\footnote{\url{http://www.worldwidewebsize.com/}, consultado el 07 de agosto del 2019.}, o la cantidad de usuarios activos diarios en redes sociales, como Facebook\footnote{\href{https://investor.fb.com/investor-news/press-release-details/2019/Facebook-Reports-First-Quarter-2019-Results/default.aspx}{https://investor.fb.com}, informe de resultados del primer trimestre del 2019 de Facebook.} con 1,56 mil millones y un crecimiento anual de un 8$\%$, o Instagram\footnote{\url{https://instagram-press.com/our-story/}, Infocenter de Instagram.} con más de mil millones de usuarios y más de 500 millones de historias diarias.

El grafo de la web representa los enlaces que tiene cada sitio hacia otro, y se modela como un grafo dirigido, con un nodo por cada sitio anexado y un arco dirigido por cada enlace que apunte a otro sitio. Los grafos de redes sociales representan relaciones entre entidades, como personas o empresas, y son modelados como grafos dirigidos o no dirigidos según corresponda. Por ejemplo, Twitter e Instagram permiten a sus usuarios seguir a otros, conectándolos de manera asimétrica, y sus representaciones corresponden a un grafo dirigido, mientras Facebook conecta de manera simétrica a sus usuarios, por tanto se representa como un grafo no dirigido.

La estructura de enlaces del grafo de la web es usada por algoritmos de ranking como \textit{PageRank} \cite{page1999pagerank}, \textit{HITS} \cite{kleinberg1999authoritative}, y \textit{Positional Power Function} \cite{herings2001measuring, herings2005positional}, como también para la detección de comunidades \cite{kumar1999trawling, flake2002self, sozio2010community}, detección de SPAM \cite{castillo2007know, becchetti2008link, saito2007large}, detección de anomalías \cite{papadimitriou2010web}, además de servir para estudiar la web y su evolución \cite{donato2005mining, kolari2004web, dourisboure2007extraction}. Los grafos de redes sociales son estudiados para reconocer actores relevantes en comunidades \cite{saito2012efficient, chen2013identifying}, identificar difundidores eficientes \cite{kitsak2010identification, lappas2010finding, zhou2014maximizing}, desarrollar algoritmos para la maximización de influencia \cite{chen2012efficient}, y estudiar cómo se propaga la información \cite{mislove2007measurement, cha2009measurement, ye2010measuring, bakshy2012role, chen2013information}.

El gran tamaño de estos grafos trae consigo grandes problemas de procesamiento. Su gran cantidad de vértices y aristas hacen prácticamente imposible mantener en memoria toda esa información, sobre mil millones de nodos y todas las conexiones entre ellos. Y procesar consultas sobre dichos grafos es muy costoso, sobre todo cuando la jerarquía de memoria de los sistemas computacionales modernos penaliza los tiempos de acceso a medida que los datos se alejan de las unidades de procesamiento.

Estos problemas han motivado a la comunidad científica a proponer estructuras comprimidas que permitan la navegación del grafo a través de consultas básicas, como obtener el listado de vecinos de un nodo. El objetivo de estas representaciones comprimidas es permitir la simulación de algoritmos de procesamiento de grafos usando mucho menos espacio en memoria que las representaciones sin comprimir.

Por otra parte, la detección de cliques maximales en grafos es un problema NP-Hard \cite{karp1972reducibility}, y existen varios enfoques para tratar el problema \cite{bron1973algorithm, eblen2012maximum, hendrix2010theoretical, bomze1999maximum, eppstein2010listing, eppstein2011listing}. Esto es particularmente de interés en grafos de red social para detectar comunidades \cite{modani2008large}, donde es provechoso contar con modelos comprimidos de dichos grafos, y que permitan obtener el listado de cliques maximales de manera rápida.

En el contexto de estructuras de datos sucintos, actualmente existen estructuras compactas que permiten representar secuencias de bits, bytes y símbolos, con soporte de consultas básicas y rápidas \cite{raman2002succinct, grossi2003high, claude2015wavelet}.

Este método primero enumera los cliques maximales y luego los representa en una estructura compacta. El esquema propuesto aprovecha tanto el tamaño como la superposición de vértices en los cliques con el objetivo de reducir el número de vértices explícitamente representados en la estructura. Los resultados muestran alto nivel de compresión y tiempos de acceso competitivo en grafos reales con un alto coeficiente de clustering y con tamaños de cliques medianos o grandes.

%En este trabajo, se propone un método de compresión de grafos poco densos y no dirigidos, que aprovecha la cantidad de vértices en común de sus cliques maximales, para crear una estructura compacta que permita responder consultas de manera eficiente. Se decide aprovechar la superposición de vértices entre cliques maximales, mientras más clusterizados sean los grafos a comprimir, es esperable que dicha superposición sea más provechosa de codificar de manera alternativa y eficiente.

%En este trabajo, se propone un método de compresión de grafos poco densos y no dirigidos, que aprovecha la cantidad de vértices en común de sus cliques maximales, para crear una estructura compacta que permita responder consultas de manera eficiente.








%%%%%%%%%%%%%%%%%%%%%%%%%%%%%%%%%%%%%%%%%%%%%%%%%%%%%%%%%%%%%%%%%%%%%%%%%%%%%%%%
%
% Paso 15: Estado del Arte
%
% 
%%%%%%%%%%%%%%%%%%%%%%%%%%%%%%%%%%%%%%%%%%%%%%%%%%%%%%%%%%%%%%%%%%%%%%%%%%%%%%%%


\chapter{MARCO TEÓRICO}\label{chap:marco}
\vskip 3.0ex

En este capítulo se presentan las definiciones de grafos, cliques maximales, y algunas métricas de clusterización, junto a codificaciones ocupadas en este trabajo.

\section{Grafos, cliques maximales y métricas de clusterización} \label{marco:grafoclique}
Se define un \textbf{grafo} $G = (V, E)$ como el conjunto finito de \textit{vértices} o \textit{nodos} $V$ y el conjunto de \textit{aristas} $E \subseteq V \times V$ (arcos). La expresión $V(G)$ representa el conjunto de sus vértices y $E(G)$ el conjunto de sus aristas. El \textit{orden} de un grafo corresponde al total de sus vértices $|V(G)|$, mientras que el \textit{tamaño} de un grafo corresponde al total de sus aristas $|E(G)|$. 

Dos vértices $v_{1}$ y $v_{2} \in V(G)$ son \textbf{adyacentes} o \textbf{vecinos} si $(v_{1}, v_{2}) \in E(G)$ y $v_{1} \neq v_{2}$.  Un grafo es \textbf{no dirigido} cuando la arista conlleva ambos sentidos, quiere decir que $(v_{1}, v_{2}) = (v_{2}, v_{1})$, ambos vértices son vecinos directos entre sí. Distintos son los grafos \textbf{dirigidos}, donde las aristas tienen un solo sentido, y $(v_{1}, v_{2}) \neq (v_{2}, v_{1})$. En este caso, $v_{2}$ es llamado \textbf{vecino directo} de $v_{1}$, mientras $v_{1}$ es llamado \textbf{vecino inverso o reverso} de $v_{2}$. Un \textbf{grafo denso} es aquel que su número de aristas es cercano al máximo. Este trabajo está enfocado a utilizar grafos no dirigidos y poco densos. 

El \textbf{grado de un vértice} $d(v)$ se define como la cantidad de vértices en $V(G)$ que son adyacentes con $v$. La \textbf{matriz de adyacencia} de un grafo $G$ corresponde a una matriz binaria cuadrada $|V(G)| \times |V(G)|$ donde una celda $(i, j)$ almacena un 1 solo si existe una arista entre los vértices que corresponden a la conjunción del par (fila, columna) $= (i, j)$. En caso contrario, la celda contiene un 0.

\begin{figure}[b]
    	\centering
    	\begin{minipage}{0.45\textwidth}
    		\centering
    		\includegraphics[scale=.3, clip,  trim=50 200 520 30]{img/graphs-bipartito.pdf}
    		
    		(a)
    	\end{minipage}
    	\begin{minipage}{0.45\textwidth}
    		\centering
    		\includegraphics[scale=.3, clip, trim=400 200 170 30]{img/graphs-bipartito.pdf}
    		
    		(b)
    	\end{minipage}

    \caption{Ejemplo de grafos bipartitos. (a) Grafo bipartito. (b) Grafo bipartito completo o biclique.}
    \label{fig:bipartito}
\end{figure}


Un grafo \textbf{k-degenerate} es un grafo no dirigido donde cada subgrafo tiene un vértice con grado a lo más \textbf{k}. El índice de \textbf{degeneracy} de un grafo, $D(G)$, es el menor valor \textbf{k} para el cual el grafo es \textbf{k-degenerate}.

Un grafo es \textbf{bipartito} cuando sus vértices se pueden separar en dos conjuntos separados $U$ y $W$, tal que se cumple $U \cup W = V$ y $U \cap W = \varnothing$. Un grafo es \textbf{bipartito completo} o \textbf{biclique}, cuando todos los vértices de un conjunto son vecinos directos de todos los vértices del otro conjunto. En la \autoref{fig:bipartito} se ilustra un ejemplo de grafo bipartito y un biclique.


Un \textbf{clique} es un subgrafo donde todos los vértices son adyacentes entre sí, es decir, $\exists V' \subseteq V(G), \forall v_{1}, v_{2} \in V', (v_{1}, v_{2}) \in E(G) $. Un \textbf{clique maximal} no puede extenderse incluyendo otro vértice adyacente, es decir, no es subconjunto de otro clique más grande. En la \autoref{fig:maxCliqueExample} se presenta ejemplo de un grafo y sus cliques maximales.

\begin{figure}
    	\centering
    	\includegraphics[width=0.5\linewidth]{img/maxCliqueExample.pdf}
    	
    \caption{Ejemplo de grafo y sus cliques maximales.}
    \label{fig:maxCliqueExample}
\end{figure}


Un \textbf{triángulo} es un subgrafo de tres vértices y tres aristas. Se define $\lambda(v)$ como la cantidad de triángulos donde participa un nodo $v$, y $\lambda(G)$ como la cantidad de triángulos de un grafo, y se calcula sumando el cálculo individual para cada vértice, y dividiendo el total en tres (por cada triángulo se cuentan 3 veces los vértices), como lo muestra la siguiente ecuación

\begin{equation}
	\lambda(G) = \dfrac{1}{3} \sum_{v \in V} \lambda(v) \label{eq:triangles}
\end{equation}

Un \textbf{triplete} es un subgrafo de tres vértices y dos aristas, donde las aristas comparten un vértice común. Se define $\tau(v)$ como la cantidad de tripletes donde $v$ es el vértice común, y $\tau(G)$ como la cantidad de tripletes de un grafo.

\begin{equation}
	\tau(G) = \sum_{v \in V} \tau(v) \label{eq:triplets}
\end{equation}

El \textbf{coeficiente de clusterización} de un vértice indica cuánto está conectado con sus vecinos, y se define como $c(v) =  \lambda(v) / \tau(v)$. El coeficiente de clusterización de un grafo ($C(G)$) es el promedio del coeficiente de todos los nodos del grafo, y se define como:

\begin{align}
	C(G) &= \dfrac{1}{|V'|} \sum_{v \in V'} c(v) \label{eq:CC} \\
	V' &= \{ v \in V | d(v) \geq 2 \} \nonumber
\end{align}

\noindent donde $V'$ es el conjunto de vértices con un grado mayor a dos. Su rango es entre $[0, 1]$, mientras más cercano a $1$ indica más conexión entre vértices.

La \textbf{transitividad} de un grafo ($T(G)$) es la probabilidad que un par de nodos adyacentes estén interconectados, y se define como:

\begin{equation}
	T(G) = \dfrac{3 \lambda(G)}{\tau(G)} \label{eq:T} 
\end{equation}

\noindent y su rango también va entre $[0, 1]$, siendo $1$ cuando todos los nodos están interconectados con todos.

Tanto el coeficiente de clusterización como la transitividad son métricas que permiten vislumbrar cuán conectados o clusterizados están los vértices de un grafo, y de sus ecuaciones se puede notar que están relacionados.

\section{Codificaciones}

Existen distintos tipos de codificaciones, según la aplicación. En esta sección se resumen algunas de relevancia para este trabajo, como algunos códigos universales o la codificación Huffman.

\subsection{Códigos universales}\label{sec:Ucoding}
Los códigos universales para enteros son un tipo de códigos que transforman enteros positivos en secuencias de bits, donde el largo de la secuencia final de bits tiene relación con el entero a codificar. Existen varios códigos de este tipo, algunos son:

\begin{itemize}
	\item \textbf{Código unario}: Se representa un entero $x$ por una secuencia de $1^{x-1}0$, donde el $0$ indica el término de la secuencia. Por ejemplo, el número $5$ se representa por la secuencia $111110$. Este código no es muy eficiente por si solo, pero se usa de base para otro tipo de códigos.
	
	\item \textbf{Código gamma ($\gamma$)}: Se representa un entero $x$ en un par concatenado de \textit{largo} y \textit{offset}. \textit{Offset} es la representación binaria de $x$, pero sin el primer 1. Por ejemplo, para $x=5$ su representación binaria es $101$, por tanto su \textit{offset} es $01$. \textit{Largo} codifica el largo de \textit{offset} en código unario. Para $x=5$, el largo de \textit{offset} es 2 bits, por tanto \textit{largo} es $110$. Concatenando ambas, el código $\gamma$ para $x=5$ es $11001$.
	
	\item \textbf{Código delta ($\delta$)}: Este código es una extensión del código $\gamma$ para enteros largos. Básicamente hacen lo mismo, pero el \textit{largo} lo representan en código $\gamma$ en vez de código unario. El código $\delta$ para $x=5$ es $10001$.
\end{itemize}

\subsection{Codificación Huffman}\label{sec:huffman}
La codificación Huffman\cite{huffman1952method} es un técnica de compresión de datos óptima que define códigos de largo variable libre de prefijos. Es óptima porque el tamaño de la representación comprimida es mínima y es libre de prefijos porque ningún código es prefijo de otro. Huffman es un algoritmo greedy basado en definir códigos mas cortos para aquellos elementos mas frecuentes. 

Una codificación binaria de largo fijo, le asigna la misma cantidad de bits a todos los símbolos por codificar. Una de largo variable le asigna menos bits a los símbolos más frecuentes, y más bits a los menos frecuentes, cuidando que las secuencias binarias cortas no sean prefijos de las más largas. La \autoref{fig:fixedVarLength} se tiene la frecuencia de seis símbolos en una secuencia de 100.000 caracteres, y se comparan ambos códigos. Para el caso de largo fijo, se requieren 3 bits por cada símbolo en la secuencia, un total de 300.000 bits. Para el caso de largo variable, el símbolo más frecuente \texttt{a} requiere un bit, y los menos frecuentes requieren 4 bits. Así, la secuencia requiere:

\begin{align*}
	(45 \cdot 1 + 13 \cdot 3 + 12 \cdot 3 + 16 \cdot 3 + 9 \cdot 4 + 5 \cdot 4) \cdot \textrm{1.000} = \textrm{224.000 bits}
\end{align*}

\noindent lo que significa una reducción cercana al $20\%$. 

Un código prefijo es aquel donde ninguna palabra codificada es usada como prefijo de otra. Estos códigos son simples de decodificar, basta con comenzar el proceso desde el primer bit hasta encontrar una de las posibles codificaciones, traducirla a su valor original, y seguir con el resto de bits codificados. Siguiendo el ejemplo, la secuencia $001011101$ se identifica como $0 \cdot 0 \cdot 101 \cdot 1101$, y se traduce como \texttt{aabe}.

Para agilizar la búsqueda, el código prefijo se puede representar como un árbol binario, donde los nodos hojas son los caracteres codificados. Siguiendo cada bit de la secuencia, se puede ir avanzando por el árbol hasta llegar a un nodo hoja, y así llegar al valor buscado. En la \autoref{fig:fixVarTrees} se ilustran los árboles de las codificaciones de la \autoref{fig:fixedVarLength}, en (a) el correspondiente a código de largo fijo, y en (b) el de largo variable.

 \begin{figure}[b]
    	\centering
    
    \begin{tabular}{l|cccccc}
    		& a & b & c & d & e & f \\
    		\midrule
    		Frecuencia (en miles) & 45 & 13 & 12 & 16 & 9 & 5 \\
    		Código largo fijo & 000 & 001 & 010 & 011 & 100 & 101 \\
    		Código largo variable & 0 & 101 & 100 & 111 & 1101 & 1100 \\
    \end{tabular}

    \caption{Ejemplo de códigos de largo fijo y largo variable.}
    \label{fig:fixedVarLength}
\end{figure}


\begin{figure}[b]
    	\centering
    	\begin{minipage}{0.45\textwidth}
    		\centering
    		\includegraphics[scale=.45, clip,  trim=20 350 350 20]{img/graphs-fixVarTrees.pdf}
    		
    		(a)
    	\end{minipage}
    	\begin{minipage}{0.45\textwidth}
    		\centering
    		\includegraphics[scale=.45, clip, trim=20 40 430 280 ]{img/graphs-fixVarTrees.pdf}
    		
    		(b)
    	\end{minipage}

    \caption{Árboles correspondientes a los códigos de la Figura~\ref{fig:fixedVarLength}. (a)~Árbol para código de largo fijo. (b)~Árbol para código de largo variable.}
    \label{fig:fixVarTrees}
\end{figure}


La codificación Huffman aprovecha todo lo anterior, utilizando una heurística \textit{greedy} para la construcción de su estructura y su compresión final. Asumiendo que se tiene una secuencia $C$ de $n$ caracteres, y que cada carácter $c \in C$ tiene una frecuencia $f[c]$ en $C$. El algoritmo crea un árbol desde los nodos hojas hacia el nodo raíz, inicialmente con $|C|$ hojas, y ejecutando $|C| - 1$ conexiones para llegar al árbol final. Luego identifica los dos elementos menos frecuentes y los conecta a un nuevo elemento, con frecuencia igual a la suma de ambos. Esto continúa hasta que todos los nodos hojas están conectados al árbol. En la \autoref{fig:huffman2} se ilustra este proceso para la secuencia ejemplo de la \autoref{fig:fixVarTrees}(b), y a continuación se detalla el proceso:

\begin{enumerate}
	\item \textbf{\autoref{fig:huffman2}(a)}: Se crean los $|C| = 6$ nodos hoja para cada carácter.
	\item \textbf{\autoref{fig:huffman2}(b)}: Se identifican los dos nodos de caracteres menos frecuentes, $f$ con $f[f] = 5$ y $e$ con $f[e] = 9$ (en miles), y se conectan a un nuevo nodo con frecuencia $5 + 9 = 14$.
	\item \textbf{\autoref{fig:huffman2}(c)}: Los siguientes nodos menos frecuentes son $c$ con $f[c] = 12$ y $b$ con $f[b] = 13$, y se conectan a otro nodo nuevo con frecuencia $12 + 13 = 25$.
	\item \textbf{\autoref{fig:huffman2}(d)}: El nodo creado que conecta $f$ con $e$ posee la menor frecuencia ($14$), y junto al nodo $d$ con $f[d] = 16$ se conectan en un nuevo nodo con frecuencia $14 + 16 = 30$.
	\item \textbf{\autoref{fig:huffman2}(e)}: Para juntar a los nuevos nodos de menor frecuencia, $25$ y $30$ respectivamente, se crea un nuevo nodo con frecuencia $25 + 30 = 55$.
	\item \textbf{\autoref{fig:huffman2}(f)}: Finalmente, se conecta el último nodo hoja restante, $a$ con $f[a] = 45$, con el reciente nodo creado de frecuencia $55$, mediante el nodo raíz con frecuencia $45 + 55 = 100$, confirmando la correcta creación del árbol.
\end{enumerate}

Se visualiza que se crearon $|C| - 1 = 6 - 1 = 5$ nodos para conectar todo el árbol. Con este resultado, se puede reconstruir de manera secuencial una secuencia codificada, simplemente recorriendo el árbol desde el nodo raíz hasta llegar a un nodo hoja, sustituir esa subsecuencia de bits por el valor de dicho nodo, y continuar con el resto de la secuencia de igual manera hasta el final. Retomando el ejemplo de secuencia $001011101$, en la \autoref{fig:huffmanBack} se visualizan los tres casos a decodificar: en (a) los primeros bits $0$ llegan al nodo hoja de \texttt{a}, en (b) la secuencia $101$ llega al nodo \texttt{b}, y en (c) la secuencia $1101$ llega al nodo \texttt{e}, dando en (d) la equivalencia entre bits y caracteres con resultado \texttt{aabe}.

\begin{figure}
    	\centering
    	\begin{minipage}{1\textwidth}
    		\centering
    		\begin{minipage}{0.45\textwidth}
    			\centering
    			\includegraphics[scale=.45, clip, trim=16 790 120 10]{img/graphs-huffman2.pdf}
    			
    			(a)
    		\end{minipage}
    		\begin{minipage}{0.45\textwidth}
    			\centering
    			\includegraphics[scale=.45, clip, trim=40 690 150 50]{img/graphs-huffman21.pdf}

    			(b)
    		\end{minipage}  		
    	\end{minipage}
    	
    	\begin{minipage}{1\textwidth}
    		\centering
    		\begin{minipage}{0.45\textwidth}
    			\centering
    			\includegraphics[scale=.4, clip, trim=10 590 120 150]{img/graphs-huffman2.pdf}
    			
    			(c)
    		\end{minipage}
    		\begin{minipage}{0.45\textwidth}
    			\centering
    			\includegraphics[scale=.4, clip, trim=10 430 160 250]{img/graphs-huffman2.pdf}
    			
    			(d)
    		\end{minipage}  
    	\end{minipage}
    	
    \begin{minipage}{1\textwidth}
    		\centering
    		\begin{minipage}{0.45\textwidth}
    			\centering
    			\includegraphics[scale=.4, clip, trim=10 210 170 410]{img/graphs-huffman2.pdf}
    			
    			(e)
    		\end{minipage}
    		\begin{minipage}{0.45\textwidth}
    			\centering
    			\includegraphics[scale=.4, clip, trim=20 40 430 280 ]{img/graphs-fixVarTrees.pdf}

    			(f)
    		\end{minipage}  
    	\end{minipage}
    	
    	 
    \caption{Etapas de construcción para codificación Huffman, para secuencia $C$ ejemplo. (a) Los 6 nodos hojas iniciales. (b) Primera unión. (c) Segunda unión. (c) Tercera unión. (d) Cuarta unión. (e) Quinta unión. (f) Última unión.}
    \label{fig:huffman2}
\end{figure}


\begin{figure}
    	\centering
    	\begin{minipage}{1\textwidth}
    		\centering
    		\begin{minipage}{0.45\textwidth}
    			\centering
    			\includegraphics[scale=.45, clip, trim=10 560 200 10]{img/graphs-huffmanBack.pdf}
    			
    			(a)
    		\end{minipage}
    		\begin{minipage}{0.45\textwidth}
    			\centering
    			\includegraphics[scale=.45, clip, trim=10 290 200 280]{img/graphs-huffmanBack.pdf}

    			(b)
    		\end{minipage}  		
    	\end{minipage}
    	
    	\begin{minipage}{1\textwidth}
    		\centering
    		\begin{minipage}{0.45\textwidth}
    			\centering
    			\includegraphics[scale=.4, clip, trim=10 20 200 550]{img/graphs-huffmanBack.pdf}
    			
    			(c)
    		\end{minipage}  
    		\begin{minipage}{0.45\textwidth}
    			\centering
			\begin{tabular}{cccc}
				0 & 0 & 101 & 1101 \\
				\midrule
				a & a & b & e \\
			\end{tabular}
		    	\vspace{5mm}
		    	
    			(d)
    		\end{minipage}  
    	\end{minipage}
    	
    	 
    \caption{Usando el árbol para decodificar Huffman. (a) Decodificando $0$. (b) Decodificando $101$. (c) Decodificando $1101$. (d) Equivalencias de bits y caracteres.}
    \label{fig:huffmanBack}
\end{figure}



\chapter{ESTADO DEL ARTE}\label{chap:background}
\vskip 3.0ex

% \documentclass statement and preamble

El problema de compresión de grafos ha sido abordado de distintas maneras en las últimas décadas. En esta sección se revisan los trabajos más relevantes del área.

\section{The WebGraph Framework, \textit{Boldi y Vigna}}
Uno de los primeros trabajos en la materia es \textit{WebGraph} de Boldi y Vigna \cite{boldi2004webgraph, boldi2004webgraph2}, apuntado a comprimir grafos dirigidos como el grafo de la Web, aprovechando la distribución potencial de las diferencias entre vecinos sucesivos, reflejados en dos características de sus enlaces ordenados por su \textit{URL}, \textbf{localidad} (hipervínculos donde sus \textit{URL} tienen un prefijo en común y si se ordenan lexicográficamente en una lista estarán muy cerca entre ellos) y \textbf{similitud} (los sitios que tienden a estar juntos en esa lista lexicográfica también tienden a tener muchos sucesores en común). Así, codifican las listas de adyacencias basadas en otras listas de adyacencias y cuán similar sean entre ellas.

Primero, cada nodo se numeran los $N$ nodos del grafo de $0$ a $N - 1$, ordenados de manera lexicográfica según sus \textit{URL}. En una primera aproximación, cada nodo tiene asociado su grado de salida (\textit{Outdegree}) y su listado de adyacencia o sucesores asociado $S(x)$. Luego,  aprovechando la localidad  de los nodos en dichas listas, se pueden representar usando las diferencias entre sus nodos, quiere decir si $S(x) = (s_{1}, s_{2}, ..., s_{k})$ es el listado de sucesores del nodo $x$ con $k$ vecinos, se codifica como $(s_{1} - x, s_{2} - s_{1} - 1, ..., s_{k} - s_{k - 1} - 1)$. En la Tabla~\ref{table:webgraph1} se muestran ambos casos, usando listado de sucesores y usando la diferencia. Para evitar tener que lidiar con números negativos, el primer número en esta nueva secuencia se codifica de la siguiente manera:

\begin{align}
	w(x) =  \begin{cases}
					2x & x \geq 0 \\
					2|x| - 1 & x < 0
				\end{cases} 
\end{align}

 \begin{table}[b]
\caption{Representación de Webgraph usando listado de sucesores directo y con brechas.}
\label{table:webgraph1}
\centering
\scriptsize

\begin{tabular}{|l|l|l|l|}
	\toprule
	Nodo & Outd. & Sucesores & Usando brechas \\
	\midrule
	... & ... & ... & ... \\
	15 & 11 & 13, 15, 16, 17, 18, 19, 23, 24, 203, 315, 1034 & 3, 1, 0, 0, 0, 0, 3, 0, 178, 111, 718 \\
	16 & 10 & 15, 16, 17, 22, 23, 24, 315, 316, 317, 3041 & 1, 0, 0, 4, 0, 0, 290, 0, 0, 2723 \\
	17 & 0 &  &  \\
	18 & 5 & 13, 15, 16, 17, 50 & 9, 1, 0, 0, 32 \\
	... & ... & ... & ... \\
\end{tabular}
\end{table} 


Avanzando en el modelo de compresión, cada nodo tiene un entero $r$ llamado referencia, si $r = 0$ la lista no está comprimida usando una referencia, y para $r > 0$ la lista $x$ está definida por la diferencia de la lista $x - r$. Un bitmap llamado \textit{copy list} codifica los sucesores que deben ser copiados a la lista, con un $1$ si el nodo referenciado esta presente en dicha lista o no. Adicionalmente se usa una lista extra para agregar todos los nodos remanentes. Las copy list se codifican en \textit{copy blocks}, donde el primer block es $0$ si la copy list comienza con un $0$. Un bloque se representa por l largo de $0$ o $1$ en la lista menos uno, y el último bloque se omite. En las Tablas \ref{table:webgraph2} y \ref{table:webgraph3} se ilustra un ejemplo para ambos casos.

\begin{table}[t]
\caption{Representación de Webgraph usando copy list.}
\label{table:webgraph2}
\centering
\scriptsize

\begin{tabular}{|l|l|l|l|l|}
	\toprule
	Nodo & Outd. & Ref. & Copy list & Nodos extra \\
	\midrule
	... & ... & ... & ... & ... \\
	15 & 11 & 0 &  & 13, 15, 16, 17, 18, 19, 23, 24, 203, 315, 1034 \\
	16 & 10 & 1 & 01110011010 & 22, 316, 317, 3041 \\
	17 & 0 &  &  &  \\
	18 & 5 & 3 & 11110000000 & 50 \\
	... & ... & ... & ... & ... \\
\end{tabular}
\end{table} 
 

\begin{table}%[t]
\caption{Representación de Webgraph usando copy blocks.}
\label{table:webgraph3}
\centering
\scriptsize

\begin{tabular}{|l|l|l|l|l|l|}
	\toprule
	Nodo & Outd. & Ref. & \# blocks & Copy blocks & Nodos extra \\
	\midrule
	... & ... & ... & ... & ... & ... \\
	15 & 11 & 0 &  &  & 13, 15, 16, 17, 18, 19, 23, 24, 203, 315, 1034 \\
	16 & 10 & 1 & 7 & 0, 0, 2, 1, 1, 0, 0 & 22, 316, 317, 3041 \\
	17 & 0 &  &  &  & \\
	18 & 5 & 3 & 1 & 4 & 50 \\
	... & ... & ... & ... & ... & ... \\
\end{tabular}
\end{table}


Como se puede apreciar de los ejemplos, la consecutividad es frecuente en el listado de nodos extra. Este hecho se puede aprovechar en un paso previo a la compresión por brecha, aislando las subsecuencias correspondientes a intervalos de enteros. Sólo los intervalos de largo no menor a un cierto umbral $L_{min}$ son considerados. Entonces, cada listado de nodos extra se comprime de la siguiente manera:

\begin{itemize}
	\item Un listado de intervalos de enteros. Se representa cada intervalo por su valor extremo izquierdo y su largo. Su valor extremo izquierdo se comprime usando la diferencia entre si mismo y el previo extremo derecho menos dos, ya que debe haber al menos un entero entre el final de un intervalo y el inicio del siguiente. Al largo del intervalo se le resta el umbral $L_{min}$.
	\item Una lista de nodos residuales, los que no son parte de los intervalos anteriores, comprimida usando la diferencia.
\end{itemize}

\begin{table}%[b]
\caption{Representación de Webgraph usando intervalos, con umbral $L_{min} = 2$.}
\label{table:webgraph4}
\centering
\scriptsize

\begin{tabular}{|l|l|l|l|l|l|l|l|l|}
	\toprule
	Nodo & Outd. & Ref. & \# blocks & Copy blocks & \# intervalos & Ext. izq. & Largo & Residuales \\
	\midrule
	... & ... & ... & ... & ... & ...  & ...  & ...  & ... \\
	15 & 11 & 0 &  &  & 2 & 0, 2 & 3, 0 & 5, 189, 111, 718 \\
	16 & 10 & 1 & 7 & 0, 0, 2, 1, 1, 0, 0 & 1 & 600 & 0 & 12, 3018 \\
	17 & 0 &  &  &  &  &  &  & \\
	18 & 5 & 3 & 1 & 4 & 0 &  &  & 50 \\
	... & ... & ... & ... & ... & ...  & ...  & ...  & ... \\
\end{tabular}
\end{table} 


Finalmente, en la Tabla~\ref{table:webgraph4} se puede apreciar la representación comprimida resultante, con un umbral $L_{min} = 2$. 

En un trabajo posterior Boldi et. al. \cite{boldi2009permuting}, usando la matriz de adyacencia y basados en aplicar permutaciones a sus filas, logran reordenar y generar una nueva matriz donde las filas, si son similares (contienen 1s en posiciones muy comunes), deben ser consecutivas o en una vecindad acotada. En otro trabajo propusieron un nuevo algoritmo llamado \textit{Layered Label Propagation} \cite{boldi2011layered} (propagación de etiquetas por capas). Su objetivo era poder ocupar las técnicas desarrolladas anteriormente para grafos de redes sociales, donde los vértices no pueden ser ordenados de manera lexicográfica. Usando la matriz de adyacencia, junto con descomponer en tareas el re-ordenamiento de la matriz y aprovechar los procesadores multi-core, logran muy buenos resultados.



\section{BFS, \textit{Apostolico y Drovandi}}
Otra alternativa de compresión bastante competitiva, también para grafos dirigidos, es la que presentan Apostolico y Drovandi \cite{apostolico2009graph}, basado en la topología del grafo de la Web en vez de las \textit{URL} subyacentes. En vez de asignarle índices a los nodos según el orden lexicográfico de sus \textit{URL}, realizan un recorrido por \textit{breath-first} o búsqueda en anchura del grafo, numerando cada nodo según el orden en que se expanden. Este proceso y su compresión inducida lo llaman \textit{Fase 1}, y la compresión de las aristas remanentes como \textit{Fase 2}.

En la Fase 1, al expandir un nodo $v_{i} \in V$ se le asignan índices enteros consecutivos a sus $k_{i}$ vecinos, y se guarda el valor $k_{i}$. Cuando el recorrido del grafo se completa, todas las aristas que pertenecen al árbol de búsqueda por anchura quedan codificadas en la secuencia $\{k_{1}, k_{2}, ..., k_{|V|}\}$ llamada \textit{traversal list} (lista de recorrido). En la Figura~\ref{fig:bfs1} se presenta un ejemplo para la Fase 1, donde en (a) se presenta el orden de los nodos asignados por BFS, y en (b) las aristas restantes junto con el listado de recorrido.

\begin{figure}%[b]
    	\centering
    	\begin{minipage}{0.45\textwidth}
    		\centering
    		\includegraphics[scale=.35, clip, trim=180 230 440 80 ]{img/arte/graphs-BFS-F1.pdf}
    		
    		(a)
    	\end{minipage}
    	\begin{minipage}{0.45\textwidth}
    		\centering
    		\includegraphics[scale=.35, clip, trim=450 230 170 80]{img/arte/graphs-BFS-F1.pdf}
    		
    		$T = \{2, 2, 1, 0, 0, 0\}$
    		
    		(b)
    	\end{minipage}

    \caption{Ejemplo de Fase 1 de BFS. (a) Índices asignados a los nodos. (b) Aristas restantes después de BFS, junto listado de recorrido $T$.}
    \label{fig:bfs1}
\end{figure}


Luego comprimen por separado trozos consecutivos de $l$ nodos, siendo $l$ un valor específico que define el nivel de compresión. Cada trozo comprimido $C$, conformado por los nodos $v_{i}, v_{i + 1}, ..., v_{i + l - 1}$, lleva prefijado la secuencia $\{k_{i}, k_{i + 1}, ..., k_{i + l - 1}\}$.

En la Fase 2, codifican la lista de adyacencia $A_{i}$ de cada nodo $v_{i} \in V$ de un trozo $C$ en orden creciente. Cada lista codificada consiste en la diferencia entre elementos adyacentes en la lista y un indicador tipo del set $\{\alpha, \beta, \chi, \phi\}$. Con $A_{i}^{j}$ indicando el elemento $j$ de la lista $A_{i}$, distinguen tres casos:

\begin{enumerate}
	\item $A_{i - 1}^{j} \leq A_{i}^{j - 1} < A_{i}^{j}$: el código es $\phi \cdot (A_{i}^{j} - A_{i}^{j - 1} - 1)$.
	\item $A_{i}^{j - 1} < A_{i - 1}^{j} \leq A_{i}^{j}$: el código es $\beta \cdot (A_{i}^{j} - A_{i - 1}^{j})$.
	\item $A_{i}^{j - 1} < A_{i}^{j} < A_{i - 1}^{j}$: se subdivide en dos subcasos:
	\begin{enumerate}
		\item Si $A_{i}^{j} - A_{i}^{j - 1} - 1 \leq A_{i - 1}^{j} - A_{i}^{j} - 1$: el código es $\alpha \cdot (A_{i}^{j} - A_{i}^{j - 1} - 1)$.
		\item De otro modo: el código es $\chi \cdot (A_{i - 1}^{j} - A_{i}^{j} - 1)$.
	\end{enumerate}
\end{enumerate}

Los tipo $\alpha$ y $\phi$ codifican la diferencia con respecto al elemento previo de la lista ($A_{i}^{j - 1}$), mientras $\beta$ y $\chi$ con respecto al elemento en la misma posición de la lista de adyacencia del nodo previo ($A_{i - 1}^{j}$). Cuando $A_{i - 1}^{j}$ no existe se reemplaza por $A_{k}^{j}$, donde $k (k < i - 1 \wedge v_{k} \in C)$ es el índice más cercano a $i$ para cual el grado de $v_{k}$ no es menor que $j$, o por un código tipo $\phi$ en caso que un nodo así no exista.

En la Tabla~\ref{table:bfs-adjacency} se ilustra un ejemplo de listado de adyacencia, y en la Tabla~\ref{table:bfs-coded} su codificación basada en los casos ya mencionados.

 \begin{table}%[b]
\caption{Lista de adyacencia para BFS, con $v_{i}$ siendo el primer nodo de un trozo.}
\label{table:bfs-adjacency}
\centering
\footnotesize

\begin{tabular}{|l|l|l|}
	\toprule
	Nodo & Grado & Adyacentes \\
	\midrule
	... & ... & ... \\
	i & 8 & 13, 15, 16, 17, 20, 21, 23, 24 \\
	i + 1 & 9 & 13, 15, 16, 17, 19, 20, 25, 31, 32 \\
	i + 2 & 0 &  \\
	i + 3 & 2 & 15, 16 \\
	... & ... & ... \\
\end{tabular}
\end{table} 


 \begin{table}%[b]
\caption{Codificación BFS del listado de adyacencia en la Tabla~\ref{table:bfs-adjacency}.}
\label{table:bfs-coded}
\centering
\footnotesize

\begin{tabular}{|l|l|l|}
	\toprule
	Nodo & Grado & Adyacentes \\
	\midrule
	... & ... & ... \\
	i & 8 & $\phi13$, $\phi1$, $\phi0$, $\phi0$, $\phi2$, $\phi0$, $\phi1$, $\phi0$ \\
	i + 1 & 9 & $\beta0$, $\beta0$, $\beta0$, $\beta0$, $\chi0$, $\alpha0$, $\beta2$, $\phi5$, $\phi0$ \\
	i + 2 & 0 &  \\
	i + 3 & 2 & $\beta2$, $\alpha0$ \\
	... & ... & ... \\
\end{tabular}
\end{table} 


Luego, aprovechan distintos tipos de redundancias en los listados de adyacencia, como se puede ver en la Tabla~\ref{table:bfs-exploting}, distinguiendo cuatro casos:

\begin{enumerate}
	\item Un conjunto de líneas idénticas (ver bloque ancho amarillo en la Tabla~\ref{table:bfs-exploting}) se codifican asignando un multiplicador a la primera línea de la secuencia.
	\item Los intervalos con grado constante de nodos (ver bloque consecutivo de 9s en la Tabla~\ref{table:bfs-exploting}), se codifican por su diferencia.
	\item Una secuencia de largo $\mathcal{L}_{min}$ de elementos idénticos (ver el bloque de $\phi1$s en la Tabla~\ref{table:bfs-exploting}), se codifica por su largo.
	\item Un bloque de filas idénticas (ver gran bloque azul en la Tabla~\ref{table:bfs-exploting}) que superen un umbral $\mathcal{A}_{min}$, se codifica por su largo.
\end{enumerate}

 \begin{table}%[b]
\caption{Ejemplo de redundancias a explotar en listado de adyacencia de BFS.}
\label{table:bfs-exploting}
\centering
\footnotesize

\begin{tabular}{|l|llllllllll|}
	\toprule
	Grado & \multicolumn{9}{l}{Adyacentes} & \\
	\midrule
	\cellcolor{blanco} ... & \multicolumn{9}{l}{\cellcolor{blanco} ...} & \\
	\cellcolor{blanco} 0 & \multicolumn{9}{l}{} & \\
	\cellcolor{rojo} 9 & \cellcolor{blanco} $\beta1,$ & \cellcolor{cafe} $\phi1,$ & \cellcolor{cafe} $\phi1,$ & \cellcolor{cafe} $\phi1,$ & \cellcolor{blanco} $\phi0,$ & \cellcolor{blanco} $\phi1,$ & \cellcolor{blanco} $\phi1,$ & \cellcolor{blanco} $\phi1,$ & \cellcolor{blanco} $\phi1,$ & \\
    \cellcolor{rojo} 9 & \cellcolor{azul} $\beta0,$ & \cellcolor{azul} $\beta1,$ & \cellcolor{azul} $\beta0,$ & \cellcolor{azul} $\beta0,$ & \cellcolor{azul} $\beta0,$ & \cellcolor{azul} $\beta0,$ & \cellcolor{azul} $\beta0,$ & \cellcolor{azul} $\beta0,$ & \cellcolor{blanco} $\beta2,$ & \\
    \cellcolor{blanco} 10 &  \cellcolor{azul} $\beta0,$ & \cellcolor{azul} $\beta1,$ & \cellcolor{azul} $\beta0,$ & \cellcolor{azul} $\beta0,$ & \cellcolor{azul} $\beta0,$ & \cellcolor{azul} $\beta0,$ & \cellcolor{azul} $\beta0,$ & \cellcolor{azul} $\beta0,$  & \cellcolor{blanco} $\beta1,$ & \cellcolor{blanco} $\phi903$ \\
    \cellcolor{blanco} 10 &  \cellcolor{azul} $\beta0,$ & \cellcolor{azul} $\beta1,$ & \cellcolor{azul} $\beta0,$ & \cellcolor{azul} $\beta0,$ & \cellcolor{azul} $\beta0,$ & \cellcolor{azul} $\beta0,$ & \cellcolor{azul} $\beta0,$ & \cellcolor{azul} $\beta0,$  & \cellcolor{blanco} $\beta223,$ & \cellcolor{blanco} $\phi900$ \\
    \cellcolor{blanco} 10 &  \cellcolor{azul} $\beta0,$ & \cellcolor{azul} $\beta1,$ & \cellcolor{azul} $\beta0,$ & \cellcolor{azul} $\beta0,$ & \cellcolor{azul} $\beta0,$ & \cellcolor{azul} $\beta0,$ & \cellcolor{azul} $\beta0,$ & \cellcolor{azul} $\beta0,$  & \cellcolor{blanco} $\beta1,$ & \cellcolor{blanco} $\alpha0$ \\
    \cellcolor{amarillo} 10 & \cellcolor{verde} $\beta0,$ & \cellcolor{verde} $\beta1,$ & \cellcolor{verde} $\beta0,$ & \cellcolor{verde} $\beta0,$ & \cellcolor{verde} $\beta0,$ & \cellcolor{verde} $\beta0,$ & \cellcolor{verde} $\beta0,$ & \cellcolor{verde} $\beta0,$ & \cellcolor{amarillo} $\beta1,$ & \cellcolor{amarillo} $\beta0$ \\
    \cellcolor{amarillo} 10 & \cellcolor{verde} $\beta0,$ & \cellcolor{verde} $\beta1,$ & \cellcolor{verde} $\beta0,$ & \cellcolor{verde} $\beta0,$ & \cellcolor{verde} $\beta0,$ & \cellcolor{verde} $\beta0,$ & \cellcolor{verde} $\beta0,$ & \cellcolor{verde} $\beta0,$ & \cellcolor{amarillo} $\beta1,$ & \cellcolor{amarillo} $\beta0$ \\
    \cellcolor{amarillo} 10 & \cellcolor{verde} $\beta0,$ & \cellcolor{verde} $\beta1,$ & \cellcolor{verde} $\beta0,$ & \cellcolor{verde} $\beta0,$ & \cellcolor{verde} $\beta0,$ & \cellcolor{verde} $\beta0,$ & \cellcolor{verde} $\beta0,$ & \cellcolor{verde} $\beta0,$ & \cellcolor{amarillo} $\beta1,$ & \cellcolor{amarillo} $\beta0$ \\
    \cellcolor{amarillo} 10 & \cellcolor{verde} $\beta0,$ & \cellcolor{verde} $\beta1,$ & \cellcolor{verde} $\beta0,$ & \cellcolor{verde} $\beta0,$ & \cellcolor{verde} $\beta0,$ & \cellcolor{verde} $\beta0,$ & \cellcolor{verde} $\beta0,$ & \cellcolor{verde} $\beta0,$ & \cellcolor{amarillo} $\beta1,$ & \cellcolor{amarillo} $\beta0$ \\
    \cellcolor{blanco} 10 &  \cellcolor{azul} $\beta0,$ & \cellcolor{azul} $\beta1,$ & \cellcolor{azul} $\beta0,$ & \cellcolor{azul} $\beta0,$ & \cellcolor{azul} $\beta0,$ & \cellcolor{azul} $\beta0,$ & \cellcolor{azul} $\beta0,$ & \cellcolor{azul} $\beta0,$  & \cellcolor{blanco} $\alpha76,$ & \cellcolor{blanco} $\alpha232$ \\
    \cellcolor{blanco} 9 &  \cellcolor{azul} $\beta0,$ & \cellcolor{azul} $\beta1,$ & \cellcolor{azul} $\beta0,$ & \cellcolor{azul} $\beta0,$ & \cellcolor{azul} $\beta0,$ & \cellcolor{azul} $\beta0,$ & \cellcolor{azul} $\beta0,$ & \cellcolor{azul} $\beta0,$  & \cellcolor{blanco} $\beta0$ & \\
	\cellcolor{blanco} ... & \multicolumn{9}{l}{\cellcolor{blanco} ...} & \\
\end{tabular}
\end{table} 


Para ello introducen un nuevo símbolo $\Sigma$, seguido de un indicador $\Sigma_{F}$ toma valor 2 si la redundancia es de tipo 3, 3 si es de tipo 4, o 1 si son ambas. Dependiendo de la redundancia a codificar, el código se expresa como `$tipo\:\Sigma\:\Sigma_{F}\:gap\:l$'  o `$tipo\:\Sigma\:\Sigma_{F}\:gap\:l\:w\:h$', siendo $tipo$ uno de los símbolos del set $\{\alpha, \beta, \chi, \phi\}$, $gap$ un entero indicando la diferencia, $l$ la cantidad de elementos idénticos en la misma línea, $w$ y $h$ el ancho y alto de las columnas y filas idénticas. En la Tabla~\ref{table:bfs-exploted} se ilustra esta codificación para el caso ejemplo de la Tabla~\ref{table:bfs-exploting}.

 \begin{table}%[b]
\caption{Ejemplo de redundancias codificadas de BFS para la Tabla~\ref{table:bfs-exploting}.}
\label{table:bfs-exploted}
\centering
\footnotesize

\begin{tabular}{|l|l|llll|}
	\toprule
	Líneas & Grado & \multicolumn{3}{l}{Enlaces} &  \\
	\midrule
	... & ... & ... & & &  \\
	0 & 0 & & & & \\
	0 & 9 & $\beta7$, & $\phi\:\Sigma\:2\:1\:1$, & $\phi0$, & $\phi\:\Sigma\:2\:1\:2$ \\
	0 & 0 & $\beta\:\Sigma\:3\:0\:7\:5$, & $\beta1$, & $\beta\:\Sigma\:2\:0\:4$, & $\beta2$ \\
	0 & 1 & $\beta1$, & $\phi903$ & & \\
	0 & 0 & $\beta223$, & $\phi900$ & & \\
	0 & 0 & $\beta1$, & $\alpha0$ & & \\
	3 & 0 & $\beta1$, & $\beta0$ & & \\
	0 & 0 & $\alpha76$, & $\alpha232$ & & \\
	0 & -1 & $\beta0$ & & & \\
	... & ... & ... & & &  \\
\end{tabular}
\end{table} 


Finalmente, usan códigos Huffman para codificar $\alpha, \beta, \chi, \phi$, y proponen una nueva codificación $\pi$ para cifrar diferencias, $\Sigma$, y otros enteros.


\section{k2-tree, \textit{Brisaboa, Ladra y Navarro}}
Una propuesta que aprovecha lo dispersa y agrupada que es la matriz de adyacencia del grafo de la Web es la propuesta por Brisaboa, Ladra y Navarro \cite{brisaboa2009k} donde proponen una estructura llamada k2-tree para ahorrar espacio y poder responder consultas tanto de vecinos directos como reversos. Esto último significa que este método, si bien fue pensado para grafos dirigidos, también se puede aplicar para no dirigidos. En un trabajo posterior, lograron mejorar sus resultados aplicando el ordenamiento por BFS antes de crear su estructura \cite{brisaboa2014compact}.

En este trabajo, proponen representar la matriz de adyacencia con un árbol $k_{2}$-nario de altura $h=\lceil(\log_k n)\rceil$, donde $n$ es el número de vértices en el grafo. Luego subdivide la matriz de adyacencia en $k^2$ submatrices de tamaño $n^{2}/k^{2}$, si una de ellas contiene solo ceros se representa solo con un bit en $0$, de lo contrario se marcan con un $1$ y se vuelven a subdividir de manera recursiva. Esta estructura soporta las consultas de vecinos directos y reversos de manera simétrica, ya que significa revisar las filas o columnas de la matriz. En la Figura~\ref{fig:k2tree} se presenta (a) un ejemplo de una matriz de adyacencia y sus subdivisiones para k2tree, y (b) un diagrama de la estructura final.

\begin{figure}%[b]
    	\centering
    	\begin{minipage}{0.45\textwidth}
    		\centering
    		\includegraphics[scale=.6, clip,  trim=270 280 250 0]{img/arte/k2-tree-matrix.pdf}
    		
    		(a)
    	\end{minipage}
    	\begin{minipage}{0.45\textwidth}
    		\centering
    		\includegraphics[scale=.5, clip, trim=50 280 410 0]{img/arte/graphs-k2tree.pdf}
    		
    		(b)
    	\end{minipage}

    \caption{Ejemplo de k2-tree. (a) Matriz de adyacencia. (b) Diagrama de estructura.}
    \label{fig:k2tree}
\end{figure}


Finalmente, la compresión se realiza representando el árbol usando dos arreglos de bits, un bitmap $T$ para representar la estructura del árbol, y un bitmap $L$ para representar las hojas, que representan las celdas de la matriz. Además usan un bitmap adicional para acelerar la resolución de consultas. 

%Una propuesta muy reciente, por Li et. al.\cite{li2019optimal} mejora considerablemente k2-tree proponiendo una manera distinta de subdividir la matriz de adyacencia.

 Francisco et al. \cite{francisco2018exploiting} discute algunos algoritmos de compresión de grafos que permiten explotar en forma amigable la computación de matrices, que además son usados en algoritmos de ranking como PageRank \cite{page1999pagerank}.


La compresión de Hernandez y Navarro \cite{hernandez2012compressed} propone una estructura compuesta por un componente que contiene los subgrafos bipartitos completos, y el resto del grafo usando otra representación comprimida existente como k2tree. Los subgrafos bipartitos completos son encontrados en el grafo original y son representados usando estructuras sucintas como wavelet trees y bitmaps comprimidos \cite{gbmp2014sea}. Esta representación comprimida proporciona resolución de consulta de vecinos directos y reversos con tiempos de acceso similares.

Una propuesta de compresión más reciente utiliza gramáticas \cite{maneth2016compressing}. Esta propuesta generaliza Re-Pair \cite{larsson2000off}, que consiste en iterativamente reemplazar los \textit{digrams} (un par de caracteres consecutivas en un string) por un nuevo símbolo hasta que no se pueden seguir reemplazando. La idea de usar gramáticas también ha sido explotada en otros enfoques. Claude y Navarro \cite{claude2010fast} proponen buscar los pares de arcos vecinos más frecuentes y reemplazarlos por nuevos símbolos en forma iterativa hasta que se cumpla un umbral, representando el grafo y el diccionario de símbolos en forma comprimida. En esta propuesta, un \textit{digram} consiste en un par de hipervínculos. Para lograr esta representación, los autores deben encontrar digrams con un número máximo de ocurrencias que no se sobrepongan (que no tengan arcos en común). Encontrar estas ocurrencias es un problema conocido como \textit{maximum matching problem} y es caro computacionalmente ($O(|V|^2\times |E|)$). Independiente de eso y al igual que k2tree, la estructura que proponen mejora con los algoritmos para ordenar los nodos en el grafo, y consiguen mejores resultados en grafos RDF con orden BFS.


El trabajo de Fisher y Peters \cite{FISCHER201639} propone un esquema de compresión que consiste en representar el grafo como un árbol donde se representan los vértices repetidos como nodos sombra (shadow nodes). La compresión efectiva de esta representación consiste en representar el árbol usando bitmaps comprimidos y los nodos sombra se representan con una secuencia de símbolos usando wavelet trees (usando la biblioteca \textit{SDSL} \cite{gbmp2014sea}). Sin embargo, esta representación ve limitado su nivel de compresión cuando los grafos a comprimir contienen muchos componentes densos, dado que en este caso la secuencia de nodos sombra puede crecer mucho.   

Entre las mejores propuestas de compresión de grafos que especialmente ofrecen accesos a vecinos directos, se encuentran la de Boldi, Rosa, Santini y Vigna \cite{boldi2011layered} y la de Grabowski y Bieniecki \cite{maneth2016compressing}. Grabowski y Bieniecki usa bloques contiguos de listas de vecinos y consigue muy buen uso de disco, pero aumenta el tiempo de acceso a vecinos directos a medida que aumenta el tamaño del bloque. Desde un punto de vista de tiempo de acceso a vecinos directos, la propuesta de Boldi y Vigna sigue siendo una de las mas rápidas. La propuesta de Hernández y Navarro \cite{tesisCecilia}, proporciona un buen compromiso entre espacio y tiempo de acceso a vecinos directos. En dicho trabajo, se propone una transformación del grafo dirigido original donde se reduce el número de arcos originales usando nodos virtuales para conectar subgrafos densos representados por grafos bipartitos completos, que incluyen cliques, y luego se apliacan distintos algoritmos de ordenamientos de vértices sobre el grafo transformado. En particular, se obtienen mejores resultados aplicando ordenamiento LLP y compresión \cite{boldi2011layered} que ordenamiento BFS (como lo reportado en \cite{Hernandez2014}).

Por otro lado, en el contexto de clustering de grafos masivos, se ha usado como heurística el método de minhashing. Por ejemplo, para encontrar subgrafos bipartitos completos y agrupar listas de adyacencia con similitud de jaccard (Buerher y Chellapilla \cite{BuehrerChellapilla}, Hernández \cite{hernandez2012compressed}).
Ambos trabajos aplican minhashing a la listas de adyacencia para crear una matriz de $n \times k$ valores de hash. Cada valor de hash en una fila de la matriz corresponde a un mapeo de la lista de adyacencia. Luego, para cada fila de la lista de adyacencia (vecinos directos de un vértice en el grafo) se aplica minhashing $k$ veces. La implementación de Buerher y Chellapilla usa $k=8$ y luego encuentra clusters buscando listas de valores hash que comparten columnas para encontrar clusters. La implementación de Hernández \cite{hernandez2012compressed} sólo usa $k=2$ para comprimir grafos. El reciente trabajo de Ertl \cite{BagMinHash} propone un algoritmo que permite expandir las posibilidades de uso a sets con peso.

Un detalle no menor en este trabajo, es la necesidad de generar el listado de cliques maximales de un grafo. Esto ha sido abordado en varios trabajos, tanto por su complejidad como por su utilidad \cite{eblen2012maximum, hendrix2010theoretical, bomze1999maximum}, y de las propuestas más recientes se destaca el trabajo de Eppstein et. al. \cite{listingcliques,  listingcliques2}, dirigido a encontrar dicho listado para grafos con matrices de adycencia ralas.





%%%%%%%%%%%%%%%%%%%%%%%%%%%%%%%%%%%%%%%%%%%%%%%%%%%%%%%%%%%%%%%%%%%%%%%%%%%%%%%%
%
% Paso 16: Cuerpo de la Tesis
% 
%%%%%%%%%%%%%%%%%%%%%%%%%%%%%%%%%%%%%%%%%%%%%%%%%%%%%%%%%%%%%%%%%%%%%%%%%%%%%%%%


\chapter{MÉTODO DE COMPRESIÓN PROPUESTO}\label{chap:clustering}
\vskip 3.0ex

En esta sección se procede a desarrollar el algoritmo de compresión de grafos dispersos, usando estructuras compactas y aprovechando la redundancia de vértices del grafo en sus cliques maximales.

EL método propuesto consiste en tres etapas. La primera consta de listar todos los cliques maximales del grafo. Luego se define una heurística eficiente para agrupar o particionar los cliques, aprovechando la superposición entre ellos. Finalmente se define una estructura compacta basada en secuencias para almacenar las particiones. 


\section{Detección de cliques maximales}

La representación del grafo mediante su grafo de cliques (ver Definición~\ref{def:cliqueGraph}) conlleva un problema, listar los cliques maximales de un grafo. Enumerarlos todos es un problema complejo desde un punto de vista teórico y práctico. 

Eppstein et. al. \cite{eppstein2010listing, eppstein2011listing} proponen un algoritmo rápido para listar cliques maximales de grafos poco densos. La complejidad de su algoritmo es $O(dn3^{d/3})$ en tiempo y $O(n+m)$ en espacio, siendo $d$ el índice de \textit{degeneracy}, $n$ la cantidad de vértices, y $m$ la cantidad de aristas del grafo (ver \autoref{sec:Cliques}).

Este algoritmo está disponible en el repositorio \textbf{Quick Cliques}\footnote{\url{https://github.com/darrenstrash/quick-cliques}}, implementado por los mismos autores. Luego, el problema se concentra en encontrar un método eficiente para dividir en particiones el grafo de cliques, que permita tanto ahorrar espacio como responder consultas sobre el grafo de manera rápida.

Con el listado de cliques maximales, se puede definir el grafo de cliques del grafo, el cual se define a continuación. 

\begin{definition} 
	\label{def:cliqueGraph}
	Gafo de cliques
	
	Dado un grafo $G = (V, E)$ y $\mathcal{C} = \{c_{1}, c_{2}, ..., c_{N} \}$ el conjunto de tamaño $N$ de cliques maximales que cubren $G$, se tiene $CG_{\mathcal{C}} = (V_{\mathcal{C}}, E_{\mathcal{C}})$ un grafo de cliques donde:
	
	\begin{enumerate}
		\item $V_{\mathcal{C}} = \mathcal{C}$
		\item $\forall c, c' \in \mathcal{C}, (c, c') \in E_{\mathcal{C}} \Longleftrightarrow c \cap c' \neq \varnothing$
	\end{enumerate}
\end{definition}

En la \autoref{fig:gafoEj} (a) se muestra un grafo no dirigido de ejemplo, en la \autoref{fig:gafoEj} (b) su listado de cliques maximales, y en la \autoref{fig:gafoEj} (c) el grafo de cliques resultante.


\begin{figure}
    	\centering
    	\begin{minipage}{0.4\textwidth}
    		\centering
    		\includegraphics[width=1\linewidth,clip=true]{img/graphs-Graph2.pdf}
    		(a)
    	\end{minipage}
    	\begin{minipage}{0.4\textwidth}
    		\centering
    		\[
	\begin{aligned}[t]
		C_{0} &: 0, 1, 2 \\
		C_{1} &: 0, 2, 3, 4 \\
		C_{2} &: 3, 5 \\
		C_{3} &: 5, 6, 7, 8, 9 \\
		C_{4} &: 4, 9
	\end{aligned}
\]

    		(b)
    	\end{minipage}
    	\begin{minipage}{0.15\textwidth}
    		\centering
    		\includegraphics[width=1\linewidth,clip=true]{img/graphs-Cliques2.pdf}
    		(c)
    	\end{minipage}
    \caption{(a) Grafo no dirigido. (b) Lista de cliques maximales. (c) Grafo de cliques.}
    \label{fig:gafoEj}
\end{figure}




\section{Particionamento del grafo de cliques}
Dado que construir el grafo de cliques maximales requiere un tiempo de computación muy alto (necesita el cómputo de intersecciones de conjuntos para todos los pares de cliques maximales), se define una heurística que estima una partición sin calcular el grafo de cliques maximales.
%Teniendo el grafo de cliques maximales, es necesario definir una heurística que permita agruparlos en particiones de manera eficiente, pensando tanto en el espacio que ocuparán las secuencias comprimidas como en tiempos de acceso secuencial y aleatorio.

Se desean encontrar particiones del grafo de cliques que exploten dicha redundancia de vértices en los cliques maximales, y permita agrupar en una misma partición a cliques que tengan una cantidad razonable de vértices en común, y los que no la tengan queden separados en otras particiones. El problema de encontrar particiones de cliques se define a continuación.

\begin{problem}
	\label{def:findPartitions}
	Encontrar particiones de cliques para el grafo de cliques $CG_{\mathcal{C}}$.
	
	Dado un grafo de cliques $CG_{\mathcal{C}} = (V_{\mathcal{C}}, E_{\mathcal{C}})$, encontrar un set de particiones de cliques $\mathcal{C}\mathcal{P} = \{cp_{1}, cp_{2}, ..., cp_{M}\}$ de $CG_{\mathcal{C}}(V_{\mathcal{C}}, E_{\mathcal{C}})$ con $M \geq 1$, tal que
	\begin{enumerate}
		%\item $\bigcup_{i \in \mathcal{C}\mathcal{P}} cp_{i} = CG_{i}$ \label{item:particiones1}
		\item $\bigcup\limits_{i = 1}^{M} cp_{i} = CG_{\mathcal{C}}$ \label{item:particiones1}
		\item $cp_{i} \cap cp_{j} = \varnothing$ para $i \neq j$ \label{item:particiones2}
		\item cualquier $cp_{i} \in \mathcal{C}\mathcal{P}$ es un subgrafo de $CG_{\mathcal{C}}(V_{\mathcal{C}}, E_{\mathcal{C}})$ inducido por el subset de vértices en $cp_{i}$ \label{item:particiones3}
	\end{enumerate}
	
\end{problem}

Esto indica que cada partición es un subgrafo del grafo de cliques maximales del grafo $G(V,E)$. El punto \ref{item:particiones2} es importante, ya que prohíbe que un clique se repita en una partición, no así un subset de vértices de grafo $G(V,E)$ que sí puede estar en varias particiones a la vez.

A continuación se plantea una heurística que, basada en el listado de cliques maximales $\mathcal{C} = \{c_{1}, c_{2}, ..., c_{N} \}$ y funciones de ranking, genere el particionamiento del grafo de cliques sin necesidad de generar dicho grafo.


\section{Algoritmo de particionamiento o clustering} \label{sec:PartitionAlgoritms}

En esta sección se procede a describir el algoritmo para generar las particiones del grafo de cliques. Para ello, en la Definición~\ref{def:rankingFunctions} se define una función de ranking, que valoriza cada vértice según ciertas características. También se detallan ciertas funciones de ranking basadas en la cantidad y tamaño de los cliques maximales donde un vértice se encuentre.

\begin{definition} 
	\label{def:rankingFunctions}
	Función de ranking
	
	Dado un grafo $G = (V, E)$ y $\mathcal{C} = \{c_{1}, c_{2}, ..., c_{N} \}$ el conjunto de tamaño $N$ de cliques maximales que cubren $G$, una función de ranking es una función $r: V \rightarrow \mathbb{R}^{+}$ que retorna un valor de puntuación para cada vértice $v \in V$.
\end{definition}

La heurística de clustering se describe en el Algoritmo~\ref{alg:clustering}. La salida del cálculo de ranking son los arreglos $D$ y $R$ (Algoritmo~\ref{alg:clustering} línea \ref{alg:clustering:rankarray}), donde $D$ contiene los índices de los cliques donde cada vértice participa en el grafo $G$, y $R$ contiene el valor de puntuación para cada vértice en $G$. La complejidad del algoritmo de cálculo de ranking se compone primero por pasar por todos los vértices en $G$ en el conjunto de cliques maximales $\mathcal{C}$, y luego ordenar $R$ de mayor a menor. La complejidad total del algoritmo es de $O(L \log L)$, donde $L=\sum_{c_i \in \mathcal{C}}|c_{i}|$.

Luego se crea un arreglo de bits $Z$ de largo $N = |\mathcal{C}|$ iniciando cada bit en cero, el cual servirá para mantener revisado si un clique ya fue incluido o no en una partición. Se recorre el arreglo $R$ y por cada vértice $u$, se obtienen los índices de los cliques donde $u$ participa según $D[u]$ y se añaden el índice $id$ de cada clique a la partición pertinente ($cpid$) solo si $Z[id] = 0$. Si el índice $id$ fue exitosamente agregado, se cambia el valor de $Z[id] = 1$. Si la partición $cpid$ contiene al menos un índice de clique, la partición es agregada a la colección $\mathcal{C}\mathcal{P}$, y se continúa procesando vértices en $R$. La complejidad del algoritmo para este paso es de $O(N+V)$. Finalmente, el algoritmo retorna la colección de particiones $\mathcal{C}\mathcal{P}$, donde cada partición contiene un set de los índices de cliques que las componen.

\begin{algorithm}
\caption{Algoritmo de particionamiento del grafo de cliques.}
\label{alg:clustering}
\begin{algorithmic}[1]
\REQUIRE $\mathcal{C}$ maximal clique collection ($N=|\mathcal{C}|$), ranking function $r(u)$
\ENSURE Returns clique-graph partition collection $\mathcal{C}\mathcal{P}$
\STATE $(D,R) \leftarrow computeRanking(r,\mathcal{C})$ (array $D$ y $R$, $\forall u \in V$ ) \label{alg:clustering:rankarray} 
\STATE Initialize bit array $Z$ of size $N$ and set each bit to 0
\FOR {$u \in R$}
    \STATE $cpid \leftarrow \emptyset$
     \FOR {$id \in D[u]$ and $Z[id]=0$} 
          \STATE $Z[id] \leftarrow 1$
          \STATE $cpid \leftarrow cpid \cup \{id\}$
    \ENDFOR
    \IF {$cpid \neq  \emptyset$}
      \STATE $\mathcal{C}\mathcal{P} \leftarrow \mathcal{C}\mathcal{P} : cpid$
    \ENDIF 
 \ENDFOR
\RETURN $\mathcal{C}\mathcal{P}$
\end{algorithmic}
\end{algorithm}


Las funciones de ranking (Definición~\ref{def:rankingFunctions}) que se proponen toman en cuenta la cantidad y el tamaño de los cliques donde participa cada vértice del grafo $G(V, E)$. Primero se define el conjunto $C(u)$ para cada vértice $u \in V$ como $C(u) = \{c \in \mathcal{C}|u \in c\}$, luego las funciones de rankings son las siguientes:

\begin{align}
	r_{f}(u) &= |C(u)| \label{eq:rankFunF}  \\ 
	r_{c}(u) &= \sum_{c \in C(u)}|c| \label{eq:rankFunC} \\ 
	r_{r}(u) &= \frac{r_{c}(u)}{r_{f}(u)} \label{eq:rankFunR}
\end{align}


La función $r_{f}(u)$ (\autoref{eq:rankFunF}) indica en cuántos cliques está presente el vértice $u$, la función $r_{c}(u)$ (\autoref{eq:rankFunC}) entrega la suma del tamaño de los cliques donde está presente el vértice $u$, y la función $r_{r}(u)$ (\autoref{eq:rankFunR}) es la razón entre $r_{c}(u)$ y $r_{f}(u)$. En la \autoref{fig:sequences} se muestra el resultado de las funciones de ranking para el caso ejemplo, y las particiones de cliques resultantes para cada una.

\begin{figure}
    \centering

    \begin{minipage}{\textwidth}
    	\centering
    	\begin{tabular}{c|c|c|c|c|c|c|c|c|c|c|}
	\cline{2-11}
	$u \in G$ & 0 & 1 & 2 & 3 & 4 & 5 & 6 & 7 & 8 & 9 \\
	\cline{2-11}
	$R_{rf}$ & 2,0 & 1,0 & 2,0 & 2,0 & 2,0 & 1,0 & 1,0 & 1,0 & 1,0 & 2,0 \\
	\cline{2-11}
	$R_{rc}$ & 7,0 & 3,0 & 7,0 & 6,0 & 6,0 & 7,0 & 5,0 & 5,0 & 5,0 & 7,0 \\
	\cline{2-11}
	$R_{rr}$ & 3,5 & 3,0 & 3,5 & 3,0 & 3,0 & 3,5 & 5,0 & 5,0 & 5,0 & 3,5 \\
    \cline{2-11}
\end{tabular}

    	
    	(a)
    \end{minipage}

    \hfill\vline\hfill
    
    \begin{minipage}{\textwidth}
    	\centering
    	\begin{tabular}{c|c|c|c|c|}
	\cline{2-5}
	$\mathcal{C}\mathcal{P}_{rf}$ & $C_{0} \quad C_{1}$ & $C_{2}$ & $C_{4}$ & $C_{3}$ \\
	\cline{2-5}
	$\mathcal{C}\mathcal{P}_{rc}$ & $C_{0} \quad C_{1}$ & $C_{2} \quad C_{3}$ & $C_{4}$ \\
	\cline{2-5}
	$\mathcal{C}\mathcal{P}_{rr}$ & $C_{3}$ & $C_{0} \quad C_{1}$ & $C_{2}$ & $C_{4}$ \\
	\cline{2-5}
\end{tabular}

    	
    	(b)
    \end{minipage}
    
    \caption{Resultados de las funciones de ranking, asociados al grafo de la \autoref{fig:gafoEj}. (a) Puntaje final. (b) Particiones de cliques.}
    \label{fig:sequences}
\end{figure}


\section{Representación en estructuras compactas}

En esta sección se detalla la estructura compacta para representar $G(V, E)$ usando las particiones $\mathcal{C}\mathcal{P}$ obtenida en la \autoref{sec:PartitionAlgoritms}. Se consideran estructuras de datos compactas  basadas en símbolos y secuencias de bits, con soporte para las operaciones de \textit{rank()}, \textit{select()} y \textit{access()}.

\subsection{Secuencias de la representación de las particiones}
La representación de las particiones consta de cuatro elementos, dos secuencias de enteros \textbf{X} e \textbf{Y}, un mapa de bits \textbf{B}, y una secuencia de bytes \textbf{BB}, las cuales se describen a continuación.

\begin{itemize}
	\item La secuencia de enteros \textbf{X} consiste en las listas concatenadas de los vértices presentes en los cliques de cada partición.
	\item El mapa de bits \textbf{B} contiene un bit por cada elemento en \textbf{X} inicializados en cero, indicando el inicio de las particiones con un uno. Además se agrega un bit extra en uno al final de la secuencia para indicar su final.
	\item La secuencia de bytes \textbf{BB} codifica en qué cliques está presente cada vértice, marcando un  $1$ en cada bit de cada byte por clique si el vértice pertenece a ese clique.
	\item La secuencia de enteros \textbf{Y} indica cuántos bytes omitir en \textbf{BB} para acceder rápidamente a la partición deseada.
\end{itemize}

La definición formal de la estructura se presenta en la Definición~\ref{def:sequences}. Se puede observar en la \autoref{eq:bbp} que $BB_{p} \in BB$ es una matriz de bytes, donde cada fila representa un vértice $u$ en $X_{p} \in X$, y las columnas corresponden a los bytes usados por los vértices para marcar los cliques donde participan en la partición. 

También se debe notar el caso especial, cuando un clique maximal queda solo en una partición, no ocupa espacio en su $BB_{p}$ correspondiente. Para poder reconocer estos casos, se agrupan al final de la estructura compacta todas estas particiones, y con esto se puede ahorrar espacio tanto en $BB$ como en $Y$.

\begin{definition} 
	\label{def:sequences}
	Representación compacta del grafo $G(V, E)$. 
	
	Dado $\mathcal{C}\mathcal{P} = \{cp_{0},...,cp_{M-1}\}$, $cp_{p} \in \mathcal{C}\mathcal{P}$, y $cp_{p}=\{c_{0},...,c_{m_{r}-1}\}$. 
	Se especifica $bpu_{p} = \ceil*{\frac{m_{r}-1}{8}}$ como la cantidad de bytes por vértice $u$ en $X_{p}$, y se definen las secuencias $X_{p}$, $B_{p}$, $BB_{p}$, $Y_{p}$ como sigue:
	
	\begin{align}
		X_{p} &= \{u \in c|c \in cp_{p}\} = \{u_{0},...u_{|X_{p}|-1}\} \\
		B_{p} &= 1:0^{|X_{p}|-1} \\
		BB_{p} &= bb[|X_{p}|][bpu_{p}]   \label{eq:bbp} \\
		bb[i][j] &= \begin{cases}
                  \sum_{k=0}^{7} 2^{k}(u_{i} \in c_{8j+k}), & bpu_{p} \neq 0  \\
                  %, & x \in c , x \in X_{p}, c \in OC_{r} \\
                  \emptyset, & otherwise \\
                 \end{cases} \nonumber \\
		Y_{p} &= \begin{cases}
				|X_{p}|\times bpu_{p} + Y_{p-1}, & bpu_{p} \neq 0  \\
				 \emptyset, & otherwise \\
			\end{cases}
	\end{align}
\end{definition}

%\textcolor{red}{Cambié la definición de Y, para dejar en claro que si una partición no tiene bytes en BB, tampoco tendrá un Y asociado.}

En la \autoref{fig:compactStructure} se presenta la estructura resultante del ejemplo, usando las particiones $\mathcal{C}\mathcal{P}_{rr}$ reordenadas. Como se puede apreciar, solo la primera partición tiene dos cliques, por tanto será la única que agregue bytes en la secuencia $BB$, codificando la pertenencia de cada clique en un bit del byte, requiriendo entonces solo un byte por vértice en $X$.

Cada secuencia $X_{p}$ consiste en la unión de los vértices que comparten todos los cliques de en la partición $p$. La secuencia $X$ está formada por la concatenación de todas las secuencias $X_{p}$. La secuencia $B$ escribe un 1 en cada inicio de una partición más uno extra para indicar el final. Para la secuencia $BB$, los cliques involucrados son $C_{0}: \{0, 1, 2\}$ y $C_{1}: \{0, 2, 3, 4\}$, ambos contienen los vértices $0$ y $2$, codificado con sus bytes en $3$, el vértice $1$ solo está presente en $C_{0}$ y se codifica con su respectivo byte en $1$, y los vértices $3$ y $4$ solo participan en $C_{1}$ y sus bytes toman el valor $2$. Finalmente la secuencia $Y$ se inicia con un  5 por la cantidad de cliques presentes en la primera partición, y como las siguientes solo tienen un clique, no se agregan más enteros.
%La secuencia $X$ se conforma por todos los vértices que conforman los cliques en cada partición, ordenados de menor a mayor. La secuencia $B$ escribe un 1 en cada inicio de una partición más uno extra para indicar el final. Para la secuencia $BB$, los cliques involucrados son $C_{0}: \{0, 1, 2\}$ y $C_{1}: \{0, 2, 3, 4\}$, ambos contienen los vértices $0$ y $2$, codificado con sus bytes en $3$, el vértice $1$ solo está presente en $C_{0}$ y se codifica con su respectivo byte en $1$, y los vértices $3$ y $4$ solo participan en $C_{1}$ y sus bytes toman el valor $2$. Finalmente la secuencia $Y$ se inicia con un  5 por la cantidad de cliques presentes en la primera partición, y como las siguientes solo tienen un clique, no se agregan más enteros.

\begin{figure}
	\centering
	\begin{minipage}{0.45\textwidth}
		\centering
		\begin{tabular}{c|c|c|c|c|}
	\cline{2-5}
	$\mathcal{C}\mathcal{P}_{rr}$ &  $C_{0} \quad C_{1}$ & $C_{3}$ & $C_{2}$ & $C_{4}$ \\
	\cline{2-5}
\end{tabular}

	
		(a)
	\end{minipage}
	\begin{minipage}{0.45\textwidth}
		\centering
		\begin{tabular}{c|c|c|c|c|c|c|c|c|c|c|c|c|c|c|c|}
	\cline{2-15}
	X: & 0 & 1 & 2 & 3 & 4 & 5 & 6 & 7 & 8 & 9 & 3 & 5 & 4 & 9 \\
	\cline{2-16}
	B: & 1 & 0 & 0 & 0 & 0 & 1 & 0 & 0 & 0 & 0 & 1 & 0 & 1 & 0 & 1 \\
	\cline{2-16}
	BB: & 3 & 1 & 3 & 2 & 2 \\
	\cline{2-6}
	Y: & 5 \\
	\cline{2-2}
\end{tabular}

		
		(b)
	\end{minipage}
	
	\caption{Ejemplo de reordenamiento y estructura compacta. (a) $\mathcal{C}\mathcal{P}_{rr}$ reordenado. (b) Estructura compacta reordenada.}
	\label{fig:compactStructure}
\end{figure}



\subsection{Algoritmos de consulta}
A continuación se presentan los algoritmos de consulta que soporta la estructura compacta. El Algoritmo~\ref{alg:sequential} reconstruye el grafo $G(V, E)$ recorriendo de manera secuencial la estructura compacta. El Algoritmo~\ref{alg:neighbors} recupera el listado de vecinos para un vértice cualquiera $u$ del grafo $G(V, E)$. El Algoritmo \ref{alg:twonodes} verifica si dos vértices son vecinos. El Algoritmo~\ref{alg:cliques} recupera el listado de cliques maximales $\mathcal{C}$ del grafo $G(V, E)$.

Para reconstruir el grafo, el Algoritmo~\ref{alg:sequential} recorre secuencialmente la estructura compacta, revisando los vecinos de cada partición. Si una partición contiene un solo clique entonces todos los vértices asociados son vecinos. Si contiene más de un clique, para cada vértice en $X$ se comparan sus bytes asociados en $BB$ con todos los demás, y si el resultado es distinto de cero, son vecinos. La cantidad de cliques se determina rápidamente al comparar el valor de la secuencia $Y$ de cada partición con la anterior; si es el mismo valor significa que hay un solo clique, si cambió es que hay más de uno.

Primero obtiene la cantidad $P$ de particiones, contando la cantidad de unos en la secuencia $B$. Para cada una de las particiones, se obtiene el índice del inicio ($s$) y final ($e$) de la partición en la secuencia $B$. Luego se calcula en $bpu_{p}$ la cantidad de bytes por vértice en la secuencia $X$, y se copia el listado de vértices de la partición actual a RAM para mejorar el tiempo de acceso. Por cada vértice, se revisan los demás restantes; si la cantidad de bytes por vértice es cero, se agregan todos los pares de vértices a la reconstrucción del grafo. De lo contrario, se comparan todos los bytes por vértice correspondientes, y si dicha comparación da algo distinto a cero, se agrega esa arista a ambos vértices involucrados, y se continúa con el siguiente vértice. Cuando se revisan todas las combinaciones de pares de vértices posibles, se prosigue con la partición siguiente. Finalmente retorna el grafo completo $G$.  La complejidad de este algoritmo es $O(P_{0} \cdot N^{2})$ cuando $bpu_{p}$ es igual a cero, de lo contrario $O(P_{1} \cdot N^{2} \cdot bpu_{p})$, siendo $P_{0}$ el número de particiones con cero bytes por vértice, $P_{1}$ las particiones que sí tienen bytes por vértice, y $N$ el largo de las particiones.

Para encontrar vecinos de vértices aleatorios, el Algoritmo~\ref{alg:neighbors}) detecta las particiones donde participa el vértice $u$ en la secuencia $X$, y luego revisa cada partición detectada. Gracias a las funciones de acceso \textit{rank()}, \textit{select()} y \textit{access()} que soporta la estructura compacta, esta tarea se realiza de manera eficiente.

Primero se cuentan las ocurrencias del vértice $u$, y por cada una se obtiene el inicio y final de las particiones donde está presente, junto con la cantidad de bytes por vértice y la copia a RAM de los vértices que tiene dicha partición. Luego, por cada vértice en la partición y posible vecino, si $bpu_{p}$ es cero se agrega directamente dicho vértice al listado de vecinos de $u$. Si no lo es, se comparan uno a uno los bytes por vértice de $u$ con su posible vecino, y si alguna comparación es distinta de cero, se agrega el vértice en evaluación al listado final y se continúa al siguiente posible. Finalmente retorna el listado de vecinos $N(u)$. La complejidad del algoritmo es $O(M_{0} \cdot N)$ cuando $bpu_{p}$ es igual a cero, y $O(M_{1} \cdot N \cdot bpu_{p})$ cuando no lo es, siendo $M_{0}$ la cantidad de particiones que contienen al vértice en la secuencia $X$ con cero bytes por vértice, $M_{1}$ el resto de particiones con bytes por vértice distinto de cero, y $N$ el largo de las particiones.

Para verificar si dos vértices son vecinos, el Algoritmo~\ref{alg:twonodes}  primero cuenta las ocurrencias de ambos nodos en la secuencia $X$, y luego revisa de manera ordenada en qué particiones se encuentra cada una de ellas. Si dos ocurrencias coinciden en una partición, revisa si existe algún bit en común entre sus correspondientes bytes de $BB$, si lo hay entonces son vecinos, de lo contrario continúa buscando otra partición donde vuelvan a encontrarse ambos nodos. La complejidad del algoritmo es $O(M_{1} + M_{2})$ cuando $bpu_{p}$ es igual a cero, y $O((M_{1} + M_{2}) \cdot bpu_{p})$ cuando no lo es, siendo $M_{1}$ el número de particiones que contienen al vértice $u_{1}$, y $M_{2}$ el número de particiones que contienen al vértice $u_{2}$.

Para recuperar el listado de cliques maximales, el Algoritmo~\ref{alg:cliques} recorre la estructura compacta de manera secuencial, y va recreando los cliques representados por los bytes en la secuencia $BB$ de cada partición.

El algoritmo primero obtiene la cantidad de particiones $P$, luego por cada una de ellas obtiene sus índices de inicio ($s$) y final ($e$) en $B$, calcula la cantidad de bytes por vértice $bpu_{p}$, y copia a RAM los vértices de la partición. Si $bpu_{p}$ es cero, quiere decir que todos los vértices pertenecen al mismo clique, por tanto los agrega como un clique directamente. Y si $bpu_{p}$ es distinto de cero, revisa por cada vértice y cada bit de cada byte la pertenencia de dicho vértice a un clique maximal; si el bit es uno lo agrega, y si es cero lo omite. Finalmente, agrega cada clique detectado al listado final de cliques maximales. La complejidad del algoritmo es $O(P_{0} \cdot N)$ cuando $bpu_{p}$ es igual a cero, y $O(P_{1} \cdot N \cdot 8 \cdot bpu_{p})$ cuando no lo es, siendo $P_{0}$ el número de particiones con cero bytes por vértice, $P_{1}$ las particiones que sí tienen bytes por vértice, y $N$ el largo de las particiones.

%\textcolor{red}{Las notaciones de complejidad de los algoritmos me causan duda, separando siempre cuando hay o no bytes en BB.}
%\textcolor{red}{Por favor revisar el algoritmo de consulta si dos nodos son vecinos. Además, realicé algunos cambios menores en los demás algoritmos, por favor revisar igual.}

\begin{algorithm}[H]
\caption{Algoritmo secuencial para reconstruir $G(V, E)$.}
\caption{Algoritmo secuencial para reconstruir $G(V, E)$.}
\label{alg:sequential}
\begin{algorithmic}[1]
    \REQUIRE $X$, $B$, $BB$, $Y$
    \ENSURE Returns $G(V, E)$

    \STATE Initialize empty graph $G$
    \STATE $P \leftarrow rank_{1}(B,|B|)$

    \FOR {$p = 1$ \TO $P$}
    	\STATE $s \leftarrow select_{1}(B, p)$
    	\STATE $e \leftarrow select_{1}(B, p + 1)$
        \STATE $bpu_{p} \leftarrow \frac{Y_{p + 1} - Y_{p}}{e - s}$
        \STATE $X_{p} \leftarrow X[s..e]$

        \FOR{$j = 0$ \TO $|X_{p}|$}
            \FOR{$k = j + 1$ \TO $|X_{p}|$}
            
            		\IF{$bpu_{p} = 0$}
                		\STATE Insert (unoriented) edges $(X_{p}[j], X_{p}[k])$ into $G$
                	\ELSE
                		\FOR{$b = 1$ \TO $bpu_{p}$}
                    		\IF {$BB_{p}[bpu_{p} \cdot j + b]$ \& $BB_{p}[bpu_{p} \cdot k + b] \neq 0$}
                        		\STATE Insert (unoriented) edge $(X_{p}[j], X_{p}[k])$ into $G$
                        		\STATE $break$
                    		\ENDIF
                		\ENDFOR
                	\ENDIF
                	
            \ENDFOR
        \ENDFOR
       
   	\ENDFOR 
    \RETURN $G$
\end{algorithmic}
\end{algorithm}


\begin{algorithm}[H]
\caption{Algoritmo para recuperar vecinos $N(u)$ de un vértice $u \in V$.}
\label{alg:neighbors}
\begin{algorithmic}[1]
    \REQUIRE $u$, $X$, $B$, $BB$, $Y$
    \ENSURE Returns $N(u)$

    \STATE Initialize empty graph $N(u)$
    \STATE $occur \leftarrow rank_{u}(X, |X|)$

    \FOR {$i = 1$ \TO $occur$}
        	\STATE $u_{p} \leftarrow select_{u}(X, i)$
        	\STATE $p \leftarrow rank_{1}(B, u_{p})$
        	\STATE $s \leftarrow select_{1}(B, p)$
    		\STATE $e \leftarrow select_{1}(B, p + 1)$
        	\STATE $bpu_{p} \leftarrow \frac{Y_{p + 1} - Y_{p}}{e - s}$
        	\STATE $X_{p} \leftarrow X[s..e]$

        	\FOR{$j = 0$ \TO $|X_{p}|$}
       		\IF{$X_{p}[j] \neq u$}
       		
       			\IF{$bpu_{p} = 0$}
            			\STATE Insert $X_{p}[j] \neq u$ to $N(u)$
            		\ELSE
            			\FOR{$b = 1$ \TO $bpu_{p}$}
                			\IF {$BB_{p}[bpu_{p} \cdot j+b]$ \& $BB_{p}[bpu_{p} \cdot (u_{p} - s) + b] \neq 0$}
                    			\STATE Insert $X_{p}[j]$ into $N(u)$
                    			\STATE $break$
                			\ENDIF
            			\ENDFOR
            		\ENDIF
            		
            	\ENDIF
        \ENDFOR
    \ENDFOR

    \RETURN $N(u)$
\end{algorithmic}
\end{algorithm}


\begin{algorithm}[H]
\caption{Algoritmo para consultar si dos nodos $u_{1}, u_{2} \in V$ son vecinos.}
\label{alg:twonodes}
\begin{algorithmic}[1]
    \REQUIRE $u_{1}$, $u_{2}$, $X$, $B$, $BB$, $Y$
    \ENSURE Returns if $(u_{1}, u_{2}) \in E$

    \STATE $occur_{1} \leftarrow rank_{u_{1}}(X, |X|)$
    \STATE $occur_{2} \leftarrow rank_{u_{2}}(X, |X|)$
    
    \STATE $ySize \leftarrow |Y|$
    
    \STATE $u1_{p} \leftarrow select_{u_{1}}(X, 1)$
    	\STATE $p1 \leftarrow rank_{1}(B, u1_{p})$
    	\STATE $i1 \leftarrow 1$

    \FOR {$i2 = 1$ \TO $occur_{2}$}
        	\STATE $u2_{p} \leftarrow select_{u_{2}}(X, i2)$
        	\STATE $p2 \leftarrow rank_{1}(B, u2_{p})$
        	
        	\WHILE {$p1 < p2$}
        		\STATE $i1 \leftarrow i1 + 1$
        		\IF {$i1 > occur_{1}$}
        			\RETURN \FALSE
        		\ENDIF
        		\STATE $u1_{p} \leftarrow select_{u_{1}}(X, i1)$
    			\STATE $p1 \leftarrow rank_{1}(B, u1_{p})$
    		\ENDWHILE
    		
    		\IF {$p1 = p2$}
    			\IF {$ySize < p1$}
    				\RETURN \TRUE
    			\ENDIF
    			
    			\STATE $s \leftarrow select_{1}(B, p1)$
    			\STATE $e \leftarrow select_{1}(B, p1 + 1)$
        		\STATE $bpu_{p} \leftarrow \frac{Y_{p1 + 1} - Y_{p1}}{e - s}$
        		
			\FOR{$b = 1$ \TO $bpu_{p}$}
                	\IF {$BB_{p}[bpu_{p}  \cdot (u1_{p} - s) + b]$ \& $BB_{p}[bpu_{p} \cdot (u2_{p} - s) + b] \neq 0$}
                		\RETURN \TRUE
                	\ENDIF
            	\ENDFOR        		
        		
    		\ENDIF
        
    \ENDFOR

    \RETURN $\FALSE$
\end{algorithmic}
\end{algorithm}


\begin{algorithm}[H]
\caption{Algoritmo para recuperar listado de cliques maximales $\mathcal{C}$ de $G(V, E)$.}
\label{alg:cliques}
\begin{algorithmic}[1]
    \REQUIRE $X$, $B$, $BB$, $Y$
    \ENSURE Returns collection of maximal cliques $\mathcal{C}$

    \STATE $\mathcal{C} \leftarrow \emptyset$
    \STATE $P \leftarrow rank_{1}(B, |B|)$

    \FOR {$p = 1$ \TO $P$}
        \STATE $s \leftarrow select_{1}(B, p)$
        \STATE $e \leftarrow select_{1}(B, p + 1)$
        \STATE $bpu_{p} \leftarrow \frac{Y_{p + 1} - Y_{p}}{e - s}$
        \STATE $X_{p} \leftarrow X[s..e]$
        
        \IF {$bpu_{p} = 0$}
            \STATE $\mathcal{C} \leftarrow \mathcal{C} \cup {X_{p}[e..s]}$
       	\ELSE
       		\STATE $BB_{p} \leftarrow \mathit{HuffmanToBytes}(BB[Y_{p}], BB[Y_{p + 1}])$

		\STATE $CC \leftarrow \emptyset$
        	\FOR{$j = 0$ \TO $|X_{p}| - 1$}
            	\STATE $cluster \leftarrow 0$
            	\STATE $iBBj \leftarrow bpu_{p} \cdot j$
            	
            	\FOR{$b = 1$ \TO $bpu_{p}$}

                	\FOR{$k = 1$ \TO $8$}
                		\STATE $CC[cluster] \leftarrow \emptyset$
                    	%\IF {($BBr[bpup*j+b]$ and $BBr[bpup*k+b]$)}
                    	\IF {$BB_{p}[iBBj + b][k] = 1$}
                        	%\STATE Insert vertex $X_{p}[j]$ to $C[cluster]$
                        	\STATE $CC[cluster] \leftarrow CC[cluster] \cup X_{p}[j]$
                    	\ENDIF
                    	\STATE $cluster \leftarrow cluster + 1$
                	\ENDFOR
                	
            	\ENDFOR
        	\ENDFOR
        \ENDIF
        
        \STATE $\mathcal{C} \leftarrow \mathcal{C} \cup \{CC[1], CC[2], \cdots, CC[cluster]\}$
    \ENDFOR

    \RETURN $\mathcal{C}$
\end{algorithmic}
\end{algorithm}




\chapter{RESULTADOS}\label{chap:contributions}
\vskip 3.0ex

En esta sección se presentan las características de los grafos $G(V, E)$ y de la estructura compacta utilizada para evaluar la propuesta, y luego se compara tanto el nivel de compresión como los tiempos de acceso secuencial y aleatorio de los algoritmos propuestos contra otros algoritmos relevantes del área.

Los algoritmos a comparar son actuales en el estado del arte para compresión de grafos, incluyendo la última versión de WebGraph \cite{boldi2011layered}, Apostolico and Drovandi \cite{apostolico2009graph}, y k2tree \cite{brisaboa2014compact}.

Todas las pruebas y experimentos se realizaron en una computadora con un procesador Intel i7 2.70GHz CPU con 12GB de RAM, y los algoritmos fueron implementados con el compilador g++ 8.2.1 con la opción de optimización O3.

\section{Grafos}
Para evaluar el rendimiento del método propuesto y comparar los resultados con el estado del arte, se seleccionan 8 grafos no densos y no dirigidos, ya que el método apunta a lograr una mejor compresión para grafos con estas características. Los 8 grafos son los siguientes:

\begin{itemize}
	\item \texttt{dblp-2010} y \texttt{dblp-2011} de WebGraph\footnote{\url{http://law.di.unimi.it/datasets.php}}.
	\item \texttt{snap-dblp} y \texttt{snap-amazon} de SNAP\footnote{\url{https://snap.stanford.edu/data/}}.
	\item \texttt{marknewman-astro} y \texttt{marknewman-condmat} de Quick-Cliques\footnote{\url{http://www.dcs.gla.ac.uk/~pat/jchoco/clique/enumeration/quick-cliques/doc/}}.
	\item \texttt{coPapersDBLP} junto a \texttt{coPapersCiteseer} de Network repository\footnote{\url{http://networkrepository.com/}}.
\end{itemize}

En la \autoref{table:gafros3} se muestran la cantidad de vértices (\boldsymbol{$|V|$}), aristas (\boldsymbol{$|E|$}), cliques maximales (\boldsymbol{$|\mathcal{C}|$}), grado medio (\boldsymbol{$\overline{d}$}) y máximo (\boldsymbol{$d_{max}$}) de los vértices de los grafos, valor de degeneracy (\boldsymbol{$D(G)$}), coeficiente de clusterización (\boldsymbol{$C(G)$}) y transitividad (\boldsymbol{$T(G)$}). De ella se pueden apreciar varias características importantes de los grafos.

En cuanto a tamaño, \texttt{marknewman-astro} y \texttt{marknewman-condmat} son los más pequeños, no superan los 50.000 vértices. Los demás tienen un tamaño bastante similar, siendo \texttt{dblp-2011} el más grande con 986.324 vértices.

Con respecto al número de aristas, \texttt{marknewman-astro} y \texttt{marknewman-condmat} también son los menores con menos de 400.000, luego \texttt{dblp-2010}, \texttt{dblp-2011}, \texttt{snap-dblp} y \texttt{snap-amazon} entre 1 y 7 millones, y finalmente \texttt{coPapersDBLP} junto a \texttt{coPapersCiteseer} con más de 30 millones de aristas.

En cantidad de cliques maximales, la mayoría tiene una cantidad proporcional a su número de vértices, a excepción de tres grafos: \texttt{snap-amazon} posee más del doble de cliques que vértices, y tanto \texttt{coPapersDBLP} y \texttt{coPapersCiteseer} tienen menos de la mitad de cliques que vértices. Esto quiere decir que su clusterización es distinta a los demás, lo que se confirma estudiando los indicadores restantes.

Con respecto al grado medio y máximo de los vértices en los grafos, se destacan \texttt{snap-dblp} con 6,62 de media pero 2.752 de máxima, que contrasta con \texttt{dblp-2011} que tiene cerca del triple de vértices, aristas y cliques, pero una media similar y un grado máximo tres veces menor. Y \texttt{coPapersDBLP} con \texttt{coPapersCiteseer} que presentan 56,41 y 73,88 de grado medio, y 3.299 con 1.188 de grado máximo, respectivamente.

Finalmente en los indicadores de clusterización, los mismos grafos \texttt{coPapersDBLP} y \texttt{coPapersCiteseer} presentan los valores más altos, lo que podría dar un indicio que los resultados de compresión y tiempos de acceso serán distintos a los demás. 


\begin{table}
	\caption{Cantidad de vértices, aristas, cliques, grado medio y máximo de los vértices, degeneracy, coeficiente de clusterización y transitividad de los grafos a comprimir.}
	\rowcolors{2}{white}{gray!10}
	\label{table:gafros3}
	\centering
	\small
	\begin{tabular}{l|r|r|r|r|r|r|r|r}
		\toprule
		Grafo & $|V|$ & $|E|$ & $|\mathcal{C}|$ & $\overline{d}$ & $d_{max}$ & $D(G)$ & $C(G)$ & $T(G)$ \\
		\midrule
		marknewman-astro & 16.706 & 242.502 & 15.794 & 14,51 & 360 & 56 & 0,66 & 0,42  \\
		marknewman-condmat & 40.421 & 351.386 & 34.274 & 8,69 & 278 & 29 & 0,64 & 0,24 \\
		dblp-2010 & 326.186 & 1.615.400 & 196.434 & 4,95 & 238 & 74 & 0,61 & 0,39  \\
        dblp-2011 & 986.324 & 6.707.236 & 806.320 & 6,80 & 979 & 118 & 0,63 & 0,20 \\
		snap-dblp & 317.080 & 2.099.732 & 257.551 & 6,62 & 2.752 & 113 & 0,63 & 0,30 \\
        snap-amazon & 403.394 & 4.886.816 & 1.023.572 & 12,11 & 343 & 10 & 0,41 & 0,16 \\
        coPapersDBLP & 540.486 & 30.491.458 & 139.340 & 56,41 & 3.299 & 336 & 0,80 & 0,65 \\
        coPapersCiteseer & 434.102 & 32.073.440 & 86.303 & 73,88 & 1.188 & 844 & 0,83 & 0,77 \\
    	\bottomrule
	\end{tabular}
\end{table}


La distribución del grado de los vértices para cada grafo se presenta en la \autoref{fig:grades}. Se puede apreciar que todos los grafos presentan una distribución similar, donde muchos vértices tienen pocos vecinos, y pocos vértices tienen muchos vecinos.

\begin{figure}
    \centering
    	\begin{minipage}{1\textwidth}
    		\centering
    		\begin{minipage}{0.45\textwidth}
    			\centering
    			\includegraphics[width=1\linewidth]{img/grades/marknewman-astro.pdf}
    			
    			(a)
    		\end{minipage}
    		\begin{minipage}{0.45\textwidth}
    			\centering
    			\includegraphics[width=1\linewidth]{img/grades/marknewman-condmat.pdf}
    			
    			(b)
    		\end{minipage}  		
    	\end{minipage}
    	
    	\begin{minipage}{1\textwidth}
    		\centering
    		\begin{minipage}{0.45\textwidth}
    			\centering
    			\includegraphics[width=1\linewidth]{img/grades/dblp-2010.pdf}
    			
    			(c)
    		\end{minipage}
    		\begin{minipage}{0.45\textwidth}
    			\centering
    			\includegraphics[width=1\linewidth]{img/grades/dblp-2011.pdf}
    			
    			(d)
    		\end{minipage}  
    	\end{minipage}
    %\caption{Distribución del grado de los vértices para cada grafo.}
    %\label{fig:grades}
%\end{figure}

%\begin{figure}%\ContinuedFloat
	%\centering
    	\begin{minipage}{1\textwidth}
    		\centering
    		\begin{minipage}{.45\textwidth}
    			\centering
    			\includegraphics[width=1\linewidth]{img/grades/snap-dblp.pdf}
    			
    			(e)
    		\end{minipage}
    		\begin{minipage}{.45\textwidth}
    			\centering
    			\includegraphics[width=1\linewidth]{img/grades/snap-amazon.pdf}
    			
    			(f)
    		\end{minipage}  
    	\end{minipage}
    	
    	\begin{minipage}{1\textwidth}
    		\centering
    		\begin{minipage}{0.45\textwidth}
    			\centering
    			\includegraphics[width=1\linewidth]{img/grades/coPapersDBLP.pdf}
    			%\includegraphics[width=1\linewidth]{img/grades/ca-coauthors.png}
    			
    			(g)
    		\end{minipage}
    		\begin{minipage}{0.45\textwidth}
    			\centering
    			\includegraphics[width=1\linewidth]{img/grades/coPapersCiteseer.pdf}
    			
    			(h)
    		\end{minipage}  
    	\end{minipage}
    	
%    	\begin{minipage}{0.45\textwidth}
%    		\centering
%    		\includegraphics[width=1\linewidth]{img/grades/ca-coauthors.png}
%    		
%    		(g)
%    	\end{minipage}  
    \caption{Distribución del grado de los vértices para cada grafo.}
    \label{fig:grades}
\end{figure}

\begin{figure}
	{\normalsize
   		\centering
    	\begin{minipage}{1\textwidth}
    		\centering
    		\begin{minipage}{0.45\textwidth}
    			\centering
    			\includegraphics[width=1\linewidth]{img/cliqueDist2/marknewman-astro.pdf}
    			
    			(a)
    		\end{minipage}
    		\begin{minipage}{0.45\textwidth}
    			\centering
    			\includegraphics[width=1\linewidth]{img/cliqueDist2/marknewman-condmat.pdf}
    			
    			(b)
    		\end{minipage}  		
    	\end{minipage}
    	
    	\begin{minipage}{1\textwidth}
    		\centering
    		\begin{minipage}{0.45\textwidth}
    			\centering
    			\includegraphics[width=1\linewidth]{img/cliqueDist2/dblp-2010.pdf}
    			
    			(c)
    		\end{minipage}
    		\begin{minipage}{0.45\textwidth}
    			\centering
    			\includegraphics[width=1\linewidth]{img/cliqueDist2/dblp-2011.pdf}
    			
    			(d)
    		\end{minipage}  
    	\end{minipage}
    	
    	\begin{minipage}{1\textwidth}
    		\centering
    		\begin{minipage}{0.45\textwidth}
    			\centering
    			\includegraphics[width=1\linewidth]{img/cliqueDist2/snap-dblp.pdf}
    			
    			(e)
    		\end{minipage}
    		\begin{minipage}{0.45\textwidth}
    			\centering
    			\includegraphics[width=1\linewidth]{img/cliqueDist2/snap-amazon.pdf}
    			
    			(f)
    		\end{minipage}  
    	\end{minipage}
    	
    	\begin{minipage}{1\textwidth}
    		\centering
    		\begin{minipage}{0.45\textwidth}
    			\centering
    			%\includegraphics[width=1\linewidth]{img/cliqueDist2/ca-coauthors.png}
    			\includegraphics[width=1\linewidth]{img/cliqueDist2/coPapersDBLP.pdf}
    			
    			(g)
    		\end{minipage}
    		\begin{minipage}{0.45\textwidth}
    			\centering
    			\includegraphics[width=1\linewidth]{img/cliqueDist2/coPapersCiteseer.pdf}
    			
    			(h)
    		\end{minipage}  
    	\end{minipage}
    	
%    	\begin{minipage}{0.45\textwidth}
%    		\centering
%    		\includegraphics[width=1\linewidth]{img/cliqueDist2/ca-coauthors.png}
%    		
%    		(g)
%    	\end{minipage}
	}
     
    \caption{Distribución del tamaño de los cliques maximales para cada grafo.}
    \label{fig:cliqueDist2}
\end{figure}


La distribución de los tamaños de los cliques maximales para cada grafo se muestra en la \autoref{fig:cliqueDist2}. Es importante notar que el gráfico del grafo \texttt{coPapersCiteseer} está truncado para efectos de comparación, pero tiene muy pocos cliques maximales por sobre el límite fijado gráficamente. Además, el grafo \texttt{snap-amazon} posee la mayor cantidad de cliques maximales pequeños.

La mayoría de los grafos contienen muchos cliques con menos de 50 vértices, a excepción de los grafos \texttt{snap-amazon}, \texttt{coPapersDBLP} y \texttt{coPapersCiteseer}. El primero contiene solo cliques pequeños, de no más de 20 vértices. Los otros dos grafos contienen una cantidad considerable de cliques de hasta 100 vértices. Esto permitirá contrastar los resultados del método propuesto entre grafos con distintas cantidades de cliques maximales grandes.



\section{Estructura compacta}

Para generar la estructura compacta se usa \texttt{SDSL}\footnote{\url{https://github.com/simongog/sdsl-lite}} desarrollada por Gog et al.\cite{gbmp2014sea}, y se implementó Huffman con acceso aleatorio \cite{huffman1952method}. Las estructuras de datos sucintas a evaluar dependen del tipo de secuencia a compactar.

\begin{itemize}
	\item Para las secuencias de símbolos, se consideran las estructuras basadas en wavelet matrix (\textit{wm}) \cite{claude2015wavelet} y wavelet tree (\textit{wt}) \cite{grossi2003high}.
	\item Para la secuencia de bits, se consideran las estructuras basadas en bitmaps comprimidos de Raman, Raman y Rao (\textit{rrr}) \cite{raman2002succinct}, y Okanohara y Sadakane (\textit{sdb}) \cite{DBLP:journals/corr/abs-cs-0610001}.
\end{itemize}

La secuencia de bytes $BB$ se comprime usando código Huffman\cite{huffman1952method}. Esto significa que $BB$ se transforma en una secuencia de bits, lo que requiere actualizar los índices de inicio de sus particiones en la secuencia $Y$, para poder obtener la secuencia de bytes equivalentes, de acuerdo a un desplazamiento en $BB$. Esto también conlleva que, por cada consulta a una partición, primero hay que decodificar todos los bytes correspondientes de esa partición antes de poder usarlos para detectar si los nodos asociados son vecinos o no. Esta codificación se implementó, ya que no es parte de la librería SDSL.

%\textcolor{red}{Creo suficiente no entrar más en detalle sobre este cambio, debido a que no es muy complejo.}

Como factores de selección, se consideran el nivel de compresión en bits, y tanto el tiempo de reconstrucción secuencial usando el Algoritmo~\ref{alg:sequential}, como el tiempo de acceso aleatorio al recuperar los vecinos de un millón de nodos, usando el Algoritmo~\ref{alg:neighbors}, para cada grafo y cada función de ranking con cada una de las estructuras antes planteadas.

%\textcolor{red}{Aquí antes mostraba las tablas con los tamaños en bytes de cada estructura compacta. Las eliminé, ya que creo no aportan en mucho comparado con la comparativa entre BPE y espacio.}

Para ello, se decide construir la estructura compacta para cada grafo y cada función de ranking, con todas las posibles combinaciones para las secuencias. En la \autoref{fig:sdslBPE2}, \autoref{fig:sdslBPE3}, \autoref{fig:sdslBPE4}, y \autoref{fig:sdslBPE5}, se compara para cada grafo el BPE de las ocho posibles combinaciones para cada función de ranking con respecto al tiempo secuencial de reconstrucción del grafo. Y en la \autoref{fig:sdslBPEAle2}, \autoref{fig:sdslBPEAle3}, \autoref{fig:sdslBPEAle4} y \autoref{fig:sdslBPEAle5}, el BPE con respecto al tiempo de acceso aleatorio al recuperar los vecinos de un millón de nodos.

%\textcolor{red}{No estoy seguro que esta sea la mejor manera de anotar tantas figuras.}

En la mayoría de los casos, la estructura que presenta la mejor relación entre BPE y tiempos es la de secuencias de símbolos con \textit{wm}, secuencia de bytes con \textit{hutu}, y secuencia de bits con \textit{sdb}.  Las opciones que presentan menores tiempos aumentan en BPE, y viceversa. Por tanto, se elige esta combinación de estructuras de secuencias como la estructura compacta a desarrollar.

\begin{figure}
    	\centering
    	\begin{minipage}{1\textwidth}
    			\centering
    			\begin{minipage}{0.8\textwidth}
    				\centering
    				\includegraphics[width=1\linewidth]{img/sdsl/aleatorioBig/marknewman-astro.pdf}
    			\end{minipage}
    			\begin{minipage}{0.15\textwidth}
    				\centering
    				\includegraphics[scale=.235, clip, trim=70 0 0 0]{img/sdsl/label.pdf}
    			\end{minipage}
    			
    			(a)		
    	\end{minipage}
    	
       	\begin{minipage}{1\textwidth}
    			\centering
    			\begin{minipage}{0.8\textwidth}
    				\centering
    				\includegraphics[width=1\linewidth]{img/sdsl/aleatorioBig/marknewman-condmat.pdf}
    			\end{minipage}
    			\begin{minipage}{0.15\textwidth}
    				\centering
    				\includegraphics[scale=.235, clip, trim=70 0 0 0]{img/sdsl/label.pdf}
    			\end{minipage}
    			
    			(b)		
    	\end{minipage}
    	
    \caption{BPE y Tiempo de acceso aleatorio medio para posibles estructuras compactas, por cada función de ranking, para los grafos marknewman-astro y marknewman-condmat.}
    \label{fig:sdslBPEAle2}
\end{figure}

\begin{figure}
    	\centering
    	\begin{minipage}{1\textwidth}
    			\centering
    			\begin{minipage}{0.8\textwidth}
    				\centering
    				\includegraphics[width=1\linewidth]{img/sdsl/aleatorioBig/dblp-2010.pdf}
    			\end{minipage}
    			\begin{minipage}{0.15\textwidth}
    				\centering
    				\includegraphics[scale=.22, clip, trim=70 0 0 0]{img/sdsl/label.pdf}
    			\end{minipage}
    			
    			(a)		
    	\end{minipage}
    	
       	\begin{minipage}{1\textwidth}
    			\centering
    			\begin{minipage}{0.8\textwidth}
    				\centering
    				\includegraphics[width=1\linewidth]{img/sdsl/aleatorioBig/dblp-2011.pdf}
    			\end{minipage}
    			\begin{minipage}{0.15\textwidth}
    				\centering
    				\includegraphics[scale=.22, clip, trim=70 0 0 0]{img/sdsl/label.pdf}
    			\end{minipage}
    			
    			(b)		
    	\end{minipage}
    	
    \caption{BPE y Tiempo de acceso aleatorio medio para posibles estructuras compactas, por cada función de ranking, para los grafos dblp-2010 y dblp-2011.}
    \label{fig:sdslBPEAle3}
\end{figure}

\begin{figure}
    	\centering
    	\begin{minipage}{1\textwidth}
    			\centering
    			\begin{minipage}{0.8\textwidth}
    				\centering
    				\includegraphics[width=1\linewidth]{img/sdsl/aleatorioBig/snap-dblp.pdf}
    			\end{minipage}
    			\begin{minipage}{0.15\textwidth}
    				\centering
    				\includegraphics[scale=.22, clip, trim=70 0 0 0]{img/sdsl/label.pdf}
    			\end{minipage}
    			
    			(a)		
    	\end{minipage}
    	
       	\begin{minipage}{1\textwidth}
    			\centering
    			\begin{minipage}{0.8\textwidth}
    				\centering
    				\includegraphics[width=1\linewidth]{img/sdsl/aleatorioBig/snap-amazon.pdf}
    			\end{minipage}
    			\begin{minipage}{0.15\textwidth}
    				\centering
    				\includegraphics[scale=.22, clip, trim=70 0 0 0]{img/sdsl/label.pdf}
    			\end{minipage}
    			
    			(b)		
    	\end{minipage}
    	
    \caption{BPE y Tiempo de acceso aleatorio medio para posibles estructuras compactas, por cada función de ranking, para los grafos snap-dblp y snap-amazon.}
    \label{fig:sdslBPEAle4}
\end{figure}

\input{figs/sdslBPEAle5}
\input{figs/sdslBPE2}
\begin{figure}
    	\centering
    	\begin{minipage}{1\textwidth}
    			\centering
    			\begin{minipage}{0.8\textwidth}
    				\centering
    				\includegraphics[width=1\linewidth]{img/sdsl/secuencialBig/dblp-2010.pdf}
    			\end{minipage}
    			\begin{minipage}{0.15\textwidth}
    				\centering
    				\includegraphics[scale=.22, clip, trim=70 0 0 0]{img/sdsl/label.pdf}
    			\end{minipage}
    			
    			(a)		
    	\end{minipage}
    	
       	\begin{minipage}{1\textwidth}
    			\centering
    			\begin{minipage}{0.8\textwidth}
    				\centering
    				\includegraphics[width=1\linewidth]{img/sdsl/secuencialBig/dblp-2011.pdf}
    			\end{minipage}
    			\begin{minipage}{0.15\textwidth}
    				\centering
    				\includegraphics[scale=.22, clip, trim=70 0 0 0]{img/sdsl/label.pdf}
    			\end{minipage}
    			
    			(b)		
    	\end{minipage}
    	
    \caption{BPE y Tiempo de acceso secuencial medio para posibles estructuras compactas, por cada función de ranking, para los grafos dblp-2010 y dblp-2011.}
    \label{fig:sdslBPE3}
\end{figure}

\input{figs/sdslBPE4}
\begin{figure}
    	\centering
    	\begin{minipage}{1\textwidth}
    			\centering
    			\begin{minipage}{0.8\textwidth}
    				\centering
    				\includegraphics[width=1\linewidth]{img/sdsl/secuencialBig/coPapersDBLP.pdf}
    			\end{minipage}
    			\begin{minipage}{0.15\textwidth}
    				\centering
    				\includegraphics[scale=.235, clip, trim=70 0 0 0]{img/sdsl/label.pdf}
    			\end{minipage}
    			
    			(a)		
    	\end{minipage}
    	
       	\begin{minipage}{1\textwidth}
    			\centering
    			\begin{minipage}{0.8\textwidth}
    				\centering
    				\includegraphics[width=1\linewidth]{img/sdsl/secuencialBig/coPapersCiteseer.pdf}
    			\end{minipage}
    			\begin{minipage}{0.15\textwidth}
    				\centering
    				\includegraphics[scale=.235, clip, trim=70 0 0 0]{img/sdsl/label.pdf}
    			\end{minipage}
    			
    			(b)		
    	\end{minipage}
    	
    \caption{BPE y Tiempo de acceso secuencial medio para posibles estructuras compactas, por cada función de ranking, para los grafos coPapersDBLP y coPapersCiteseer.}
    \label{fig:sdslBPE5}
\end{figure}



\section{Comparación de funciones de ranking}

A continuación se comparan las estructuras compactas resultantes, usando las tres funciones de ranking basadas en la frecuencia del vértice en los cliques maximales $r_{f}(u)$, en la cantidad de vecinos en los cliques del vértice $r_{c}(u)$, y la razón entre ambas funciones $r_{r}(u)$. 

En la \autoref{fig:bpe3} y la \autoref{table:bpe3} se muestran los BPE de las estructuras compactas finales, aplicando las funciones de ranking en la heurística de clusterización. Se puede apreciar que $r_{f}(u)$ logra los mejores resultados de compresión, con excepción del grafo \texttt{snap-amazon} donde $r_{r}(u)$ tiene un BPE levemente menor. Si este parámetro fuera el único a considerar para elegir la mejor función, $r_{f}(u)$ sería la mejor opción, pero es necesario profundizar en la conformación de la estructura.

\begin{figure}
    	\centering
    	\includegraphics[width=1\linewidth]{img/bpe3.pdf}
    	
    \caption{BPE de las estructuras compactas para las funciones de ranking.}
    \label{fig:bpe3}
\end{figure}

\begin{table}
	\caption{Comparativa de BPE de las estructuras compactas para las funciones de ranking.}
	\rowcolors{2}{white}{gray!10}
	\label{table:bpe3}
	\centering
	\begin{tabular}{l|r|r|r}
		\toprule
		Grafo & $r_{r}$ & $r_{c}$ & $r_{f}$\\
		\midrule
		marknewman-astro & 3.96 & 3.86 & 3.82 \\
		marknewman-condmat & 5.74 & 5.47 & 5.44 \\
		dblp-2010 & 6,27 & 7,45 & 8,00 \\
        dblp-2011 & 6,87 & 10,18 & 11,37 \\
		snap-dblp & 7,19 & 11,21 & 9,92 \\
        snap-amazon & 10,49 & 15,66 & 12,33 \\
        coPapersDBLP & 6,87 & 10,18 & 11,37 \\
        coPapersCiteseer & 0,80 & 3,29 & 1,83 \\
        \bottomrule
	\end{tabular}
\end{table}


En la \autoref{fig:nPartitions} se presentan la cantidad de particiones que contiene las estructuras compactas para cada función de ranking. Se aprecia con bastante claridad que la función $r_{r}(u)$ genera más particiones que las otras dos, y dado que el BPE expuesto en la \autoref{fig:bpe3} no refleja esta diferencia, se puede intuir que las particiones son más pequeñas. Para profundizar en este punto, se estudiará la composición de las secuencias de las estructuras compactas para cada función de ranking.

\begin{figure}
    	\centering
    	\includegraphics[width=1\linewidth]{img/Npartitions3.pdf}
    	
    \caption{Número de particiones en las estructuras compactas para las funciones de ranking.}
    \label{fig:nPartitions}
\end{figure}


En la \autoref{fig:proportionBits}(a) se ilustra la proporción de bits para cada secuencia dentro de la estructura compacta, para cada función de ranking. En la \autoref{fig:proportionBits}(b) se ilustra la misma proporción normalizada. Como se puede observar, la secuencia que más aporta para todos los casos es la de vértices \textit{X}, seguida de la secuencia de bytes \textit{BB}, luego \textit{Y} y finalmente la secuencia de bits \textit{B}. Nuevamente las funciones de ranking $r_{f}(u)$ y $r_{c}(u)$ tienen resultados similares, y para la función $r_{r}(u)$ la secuencia \textit{X} aumenta su proporción mientras que \textit{BB} disminuye. Esto seguramente afectará positivamente el tiempo de respuesta de los algoritmos propuestos, ya que todos requieren comparar los bytes de \textit{BB} de cada partición entre ellos, y si hay menos bytes requerirá menos tiempo.

\begin{figure}
    	\centering
    	\begin{minipage}{1\textwidth}
    		\centering
    		\includegraphics[width=1\linewidth]{img/bits.pdf}
    		
    		(a)
    	\end{minipage}  
    	\begin{minipage}{1\textwidth}
    		\centering
    		\includegraphics[width=1\linewidth]{img/bitsNorm.pdf}
    		
    		(b)
    	\end{minipage}  
    \caption{(a) Proporción de bits por cada secuencia en la estructura compacta, para cada función de ranking. (b) Proporción normalizada.}
    \label{fig:proportionBits}
\end{figure}


Para estudiar la composición de las particiones para cada función de ranking, en la \autoref{fig:maxNodes} se ilustran la cantidad máxima de vértices en la secuencia \textit{X} por partición, y en la \autoref{fig:maxBytes} la cantidad máxima de bytes por vértice en la secuencia \textit{BB} de las estructuras compactas resultantes. Como se puede apreciar, la función $r_{r}(u)$ presenta consistentemente los menores valores entre las tres funciones, lo que permite asegurar que es la que mejor agrupa y usa el espacio de las particiones en la estructura compacta.

\begin{figure}
    	\centering
    	\includegraphics[width=1\linewidth]{img/maxNodes.pdf}
    	
    \caption{Número máximo de vértices en la secuencia $X$ para las funciones de ranking.}
    \label{fig:maxNodes}
\end{figure}

\begin{figure}
    	\centering
    	\includegraphics[width=1\linewidth]{img/maxBytes.pdf}
    	
    \caption{Número máximo de bytes por nodo para las funciones de ranking.}
    \label{fig:maxBytes}
\end{figure}


En la \autoref{fig:cdfBPN} se puede estudiar la función de distribución acumulativa (CDF) para la cantidad de bytes por vértice en la estructura compacta, para cada función de ranking. Se confirma que para la función $r_{r}(u)$ las particiones contienen menos bytes en la secuencia \textit{BB}, ya que la cantidad de bytes por vértice es significativamente menor.

\begin{figure}
    	\centering
    	\begin{minipage}{1\textwidth}
    		\centering
    		\begin{minipage}{0.45\textwidth}
    			\centering
    			\includegraphics[width=1\linewidth]{img/cdf/marknewman-astro.pdf}
    			
    			(a)
    		\end{minipage}
    		\begin{minipage}{0.45\textwidth}
    			\centering
    			\includegraphics[width=1\linewidth]{img/cdf/marknewman-condmat.pdf}
    			
    			(b)
    		\end{minipage}  		
    	\end{minipage}
    	
    	\begin{minipage}{1\textwidth}
    		\centering
    		\begin{minipage}{0.45\textwidth}
    			\centering
    			\includegraphics[width=1\linewidth]{img/cdf/dblp-2010.pdf}
    			
    			(c)
    		\end{minipage}
    		\begin{minipage}{0.45\textwidth}
    			\centering
    			\includegraphics[width=1\linewidth]{img/cdf/dblp-2011.pdf}
    			
    			(d)
    		\end{minipage}  
    	\end{minipage}
    	
    	\begin{minipage}{1\textwidth}
    		\centering
    		\begin{minipage}{0.45\textwidth}
    			\centering
    			\includegraphics[width=1\linewidth]{img/cdf/snap-dblp.pdf}
    			
    			(e)
    		\end{minipage}
    		\begin{minipage}{0.45\textwidth}
    			\centering
    			\includegraphics[width=1\linewidth]{img/cdf/snap-amazon.pdf}
    			
    			(f)
    		\end{minipage}  
    	\end{minipage}
    	
    	\begin{minipage}{1\textwidth}
    		\centering
    		\begin{minipage}{0.45\textwidth}
    			\centering
    			\includegraphics[width=1\linewidth]{img/cdf/coPapersDBLP.pdf}
    			
    			(g)
    		\end{minipage}
    		\begin{minipage}{0.45\textwidth}
    			\centering
    			\includegraphics[width=1\linewidth]{img/cdf/coPapersCiteseer.pdf}
    			
    			(h)
    		\end{minipage}  
    	\end{minipage}
    	
%    	\begin{minipage}{0.45\textwidth}
%    		\centering
%    		\includegraphics[width=1\linewidth]{img/cdf/bytesPerNode/ca-coauthors.png}
%    		
%    		(g)
%    	\end{minipage}  
    \caption{CDF para bytes por vértice en estructuras compactas para cada función de ranking.}
    \label{fig:cdfBPN}
\end{figure}



En la \autoref{fig:timesRanking} se ven los tiempos de acceso aleatorio para la obtención de vecinos de cualquier vértice $u \in G(V, E)$ desde las estructuras compactas generadas con las tres funciones de ranking en comparación. Como se esperaba, los tiempos menores se obtienen usando la estructura basada en la función $r_{r}(u)$, ya que no tiene que comparar tantos bytes por particiones como las otras dos.

\begin{figure}
    	\centering
    	\includegraphics[width=1\linewidth]{img/timesRanking.pdf}
    	
    \caption{Tiempos de acceso aleatorio para las funciones de ranking.}
    \label{fig:timesRanking}
\end{figure}


Entonces, considerando la gran ventaja en tiempo de acceso y la leve diferencia en compresión, se concluye que la mejor alternativa entre las tres funciones de ranking es $r_{r}(u)$. 

En la \autoref{table:constructTimes} se muestran los tiempos en segundos de la generación del listado de cliques maximales $\mathcal{C}$ ($t_{\mathcal{C}}$) directo del grafo, el tiempo de generar la estructura compacta desde el listado de cliques ($t_{CS}$), el tiempo total de generar la estructura compacta ($t_{T} = t_{\mathcal{C}} + t_{CS}$) y el tiempo para recuperar el listado de cliques $\mathcal{C}$ desde la estructura compacta ($t'_{\mathcal{C}})$) usando el Algoritmo~\ref{alg:cliques}. 

Se debe notar que el tiempo para recuperar el listado de cliques desde la estructura compacta es menor al requerido desde el grafo directamente. Si bien se puede argumentar que para llegar a la estructura compacta se debe generar el listado desde el grafo, por tanto $t_{\mathcal{C}}$ es necesario pagarlo ineludiblemente, una vez generada la estructura se puede obtener $\mathcal{C}$ de ella, en menor tiempo y sin tener que descomprimir el grafo. Se destaca este contraste en los grafos \texttt{coPapersDBLP} y \texttt{coPapersCiteseer}, donde desde la estructura compacta es sobre 10 veces más rápida.

\begin{table}
	\caption{Tiempos de obtención de listado de cliques maximales y construcción de la estructura compacta, en segundos.}
	\rowcolors{2}{white}{gray!10}
	\label{table:constructTimes}
	\centering
	\begin{tabular}{l|r|r|r|r}
		\toprule
		Grafo & $t_{\mathcal{C}}$ & $t_{CS}$ & $t_{T}$ & $t'_{\mathcal{C}}$ \\
		\midrule
		marknewman-astro & 0,18 & 0,28 & 0,46 & 0,05  \\
		marknewman-condmat & 0,28 & 0,40 & 0,68 & 0,11 \\
		dblp-2010 & 1,12 & 1,46 & 2,58 & 0,57  \\
         dblp-2011 & 5,58 & 7,30 & 12,88 & 2,77 \\
		snap-dblp & 1,68 & 2,30 & 3,98 & 0,86 \\
         snap-amazon & 5,93 & 8,44 & 14,37 & 3,01 \\
         coPapersDBLP & 17,96 & 3,44 & 21,40 & 1,39 \\
         coPapersCiteseer & 26,70 & 4,70 & 31,40 & 0,96 \\
         \bottomrule
	\end{tabular}
\end{table}

A continuación se procede a comparar la opción de compresión seleccionada con los algoritmos del estado del arte ya mencionados.




\section{Comparando con estado del arte}
En esta sección se compara el nivel de compresión y los tiempos de acceso de la estructura compacta usando la función de ranking $r_{r}(u)$ seleccionada en la sección anterior, con los algoritmos más recientes de WebGraph \cite{boldi2011layered}, Apostolico and Drovandi (AD) \cite{apostolico2009graph}, y k2tree \cite{brisaboa2009k, brisaboa2014compact}.

A continuación se detallan las notaciones a usar en el resto de la sección.

\begin{itemize}
	\item Para la estructura compacta propuesta, basada en las superposición de cliques maximales, se diferencian dos casos:
		\begin{itemize}
			\item $C_{rf}$: Usando la estructura con función de ranking $r_{f}(u)$.
			\item $C_{rr}$: Usando la estructura con función de ranking $r_{r}(u)$.
		\end{itemize}			
	\item En el caso de Webgraph, como el algoritmo genera una estructura adicional para el caso de acceso aleatorio, se diferencian dos casos: 
		\begin{itemize}
			\item $WG_{s}$: Para el caso de acceso secuencial. 
			\item $WG_{a}$: Para el caso de acceso aleatorio.
		\end{itemize}
	\item Para el caso de k2tree, se diferencian dos casos:
		\begin{itemize}
			\item $k2T$: Cuando el algoritmo usa el orden del grafo original.
			\item $k2T_{BFS}$: Cuando el algoritmo usa el orden por BFS.
		\end{itemize}			
	\item $AD$: El algoritmo BFS de Apostolico y Drovandi.
\end{itemize}

Es importante recordar que los algoritmos Webgraph y AD están orientados a comprimir grafos dirigidos. Esto requiere que cada arco de los grafos no dirigidos a evaluar deben ser anotados en ambos sentidos antes de ser comprimidos. 

Mención especial requiere k2-tree, donde la autora proporcionó una versión mejorada del algoritmo orientado específicamente a grafos no dirigidos, que reduce el espacio necesario para representar el grafo considerando la mitad de la matriz de adyacencia, junto con la capacidad de dicho modelo de entregar los listados de vecinos directos como reversos. Para obtener el listado de adyacencia de un nodo, se debe obtener ambos listados y retornar su unión.

Con esto presente, en la \autoref{table:BPEcomp} se comparan los BPE de todos los casos, resaltando los mejores resultados. Para dos de los grafos comprimidos, \texttt{marknewman-astro} y \texttt{coPapersDBLP}, la propuesta usando la función de ranking $r_{f}(u)$ es la que obtiene el menor resultado, seguido muy de cerca por la versión usando la función de ranking $r_{r}(u)$. Para \texttt{marknewman-condmat} también logra el mejor resultado, pero seguido por k2tree con BFS. Con el grafo \texttt{coPapersCiteseer} se obtiene un resultado muy cercano al mejor caso de k2tree con BFS, y en los demás, los dos mejores resultados los obtienen alguna de las dos versiones de k2-tree, seguido siempre por alguna de las dos opciones del método propuesto. Tanto Webgraph como AD no logran competir en compresión con los demás.

Esto confirma que la propuesta es competitiva con el estado del arte, solo considerando el nivel de compresión. A continuación se evalúa el rendimiento en tiempos de acceso.

\begin{table}
	\caption{BPE de algoritmos de compresión.}
	\rowcolors{2}{white}{gray!10}
	\label{table:BPEcomp}
	\centering
	\begin{tabular}{l|r|r|r|r|r|r}
		\toprule
		Grafo & $clique_{rr}$ & $k2tree$ & $k2tree_{BFS}$ & $AD$ & $WG_{a}$ & $WG_{s}$\\
		\midrule
		marknewman-astro & \textbf{4,45} & 9,28 & 8,05 & 5,67 & 8,10 & 7,30\\
		marknewman-condmat & \textbf{6,20} & 12,06 & 10,43 & 7,86 & 11,78 & 10,45\\
		dblp-2010 & \textbf{6,28} & 7,45 & 8,00 & 6,71 & 8,67 & 6,91 \\
        dblp-2011 & \textbf{6,87} & 10,18 & 11,37 & 9,67 & 10,13 & 8,71 \\
		snap-dblp & \textbf{7,19} & 11,21 & 9,92 & 8,14 & 11,80 & 10,17 \\
        snap-amazon & \textbf{10,49} & 15,66 & 12,33 & 10,96 & 14,50 & 13,35 \\
        coPapersDBLP & \textbf{0,80} & 3,29 & 1,83 & 1,81 & 2,71 & 2,48 \\
        coPapersCiteseer & \textbf{0,52} & 2,35 & 0,87 & 0,85 & 1,79 &  1,63 \\
		\bottomrule
	\end{tabular}
\end{table}


Para comparar el tiempo de acceso aleatorio, se prueba el Algoritmo~\ref{alg:neighbors} recuperando los vecinos de un millón de vértices aleatorios de $G(V, E)$, y se divide el tiempo que demora dicha solicitud por la cantidad de aristas recuperadas. En la \autoref{table:timesRandom} se presentan los resultados, sin considerar el caso de Webgraph secuencial ($WG_{s}$), ya que solo se considera acceso aleatorio. 

Como se puede apreciar, Webgraph presenta los menores tiempos de acceso entre todos los algoritmos, y AD lo sigue siempre en segundo lugar. Luego para los grafos \texttt{dblp-2010}, \texttt{dblp-2011}, \texttt{snap-dblp} y \texttt{snap-amazon}, el método propuesto logra mejores resultados que ambas versiones de k2-tree, con especial atención a \texttt{dblp-2011} donde logra ser el doble de rápido. Para el resto de los casos, el resultado es bastante competitivo entre esos métodos.

\begin{table}
	\caption{Tiempos de acceso aleatorio, en microsegundos por arco.}
	\rowcolors{2}{white}{gray!10}
	\label{table:timesRandom}
	\centering
	\begin{tabular}{l|r|r|r|r|r}
		\toprule
		Grafo & $clique_{rr}$ & $k2tree$ & $k2tree_{BFS}$ & $AD$ & $WG_{a}$ \\
        \midrule    
        marknewman-astro & 2,67 & 2,58 & 1,33 & 1,79 & \textbf{0,052} \\
        marknewman-condmat & 3,16 & 5,53 & 2,81 & 2,32 & \textbf{0,063} \\
        dblp-2010 & 3,70 & 5,55 & 4,84 & 2,15 & \textbf{0,097} \\
        dblp-2011 & 4,66 & 11,43 & 10,69 & 2,36 & \textbf{0,114} \\
        snap-dblp & 4,07 & 10,35 & 6,93 & 2,30 & \textbf{0,125} \\
        snap-amazon & 6,99 & 13,97 & 7,13 & 2,47 & \textbf{0,087} \\
        coPapersDBLP & 1,51 & 1,89 & 1,16 & 0,73 & \textbf{0,045} \\
        coPapersCiteseer & 1,30 & 0,95 & 0,50 & 0,45 & \textbf{0.037} \\
        \bottomrule
	\end{tabular}
\end{table}


El tiempo de reconstrucción secuencial se midió usando el Algoritmo~\ref{alg:sequential}. En la \autoref{table:timesSecuencial} se presentan los resultados obtenidos para los algoritmos evaluados. En este caso, si bien el método propuesto es más lento que los demás, para casos como \texttt{marknweman-astro}, \texttt{marknewman-condmat}, \texttt{dblp-2010} y \texttt{snap-dblp}, la diferencia es menor a un segundo. Para el resto de los casos, los grafos tienen una cantidad considerable de arcos, \texttt{dblp-2011} y \texttt{snap-amazon} tienen la mayor cantidad de cliques, y \texttt{coPapersDBLP} con \texttt{coPapersCiteseer} tienen muchos cliques de hasta 100 vértices, y presentan una relación no proporcional entre cantidad de vértices y cantidad de cliques, como lo muestra la \autoref{table:gafros3}, lo que afecta bastante a la hora de recuperar dichos grafos.

Para evaluar mejor la competitividad, en las Figuras~\ref{fig:bpetAle1}~-~\ref{fig:bpetAle4} se muestras los BPE con respecto a los tiempo de acceso aleatorio en micro-segundos, y en las Figuras~\ref{fig:bpetSec1}~-~\ref{fig:bpetSec4} los BPE con respecto a los tiempos de reconstrucción secuencial en segundos, obtenidos para cada algoritmo y cada grafo.

Se puede apreciar que, para el caso aleatorio, si bien el método de compresión se ubica casi siempre en el cuadrante de menor BPE y mayor tiempo, otros logran una ubicación de menor calidad, como los casos de los algoritmos de k2tree para los grafos \texttt{dblp-2010}, \texttt{dbpl-2011} en la \autoref{fig:bpetAle2}, y \texttt{snap-dblp} en la \autoref{fig:bpetAle3}. En el caso secuencial, para el set de grafos pequeños \texttt{marknewman-astro} y \texttt{marknewman-condmat}, se logra muy buena ubicación en el cuadrante de menor BPE y menor tiempo (\autoref{fig:bpetSec1}), pero para los demás grafos el tiempo es muy lejano a los demás algoritmos. 

En cuanto a verificar si dos nodos son vecinos o no, la estructura planteada es la única que tiene dicha consulta implementada y no requiere descomprimir para responder directamente. Tanto Webgraph como AD, al usar diferencias para codificar los listados de adyacencia, primero requieren obtener el listado de vecinos de un nodo y luego revisar si el segundo nodo pertenece o no a dicha lista. La propuesta de k2tree podría responder dicha consulta, ya que codifica la matriz de adyacencia de tal manera que podría revisar directamente la vecindad de dos nodos, pero no la tiene implementada.

\begin{table}
	\caption{Tiempos de reconstrucción secuencial del grafo, en segundos.}
	\rowcolors{2}{white}{gray!10}
	\label{table:timesSecuencial}
	\centering
	\begin{tabular}{l|r|r|r|r|r}
		\toprule
		Grafo & $C_{rf}$ & $C_{rr}$ & $k2T$ & $k2T_{BFS}$ & $WG_{s}$ \\
        \midrule
        marknewman-astro & 0,09 & 0,09 & 0,03 & \textbf{0,02} & 0,28 \\
        marknewman-condmat & 0,16 & 0,16 & 0,07 & \textbf{0,04} & 0,52 \\
        dblp-2010 & 0,79 & 0,82 & 0,18 & \textbf{0,16} & 1,09 \\
        dblp-2011 & 4,61 & 4,45 & \textbf{1,10} & 1,31 & 2,41 \\
        snap-dblp & 1,16 & 1,26 & 0,58 & \textbf{0,35} & 1,20 \\
        snap-amazon & 7,09 & 4,53 & 1,36 & \textbf{1,13} & 1,30 \\
        coPapersDBLP & 5,68 & 5,81 & 1,45 & \textbf{1,01} & 1,59 \\
        coPapersCiteseer & 4,62 & 5,46 & 1,33 & \textbf{0,65} & 1,56 \\
        \bottomrule
	\end{tabular}
\end{table}

 

\begin{figure}
    	\centering
    	\begin{minipage}{1\textwidth}
    			\centering
    			\begin{minipage}{0.8\textwidth}
    				\centering
    				\includegraphics[width=1\linewidth]{img/bpeTimes/aleatorio/marknewman-astro.pdf}
    			\end{minipage}
    			\begin{minipage}{0.15\textwidth}
    				\centering
    				\includegraphics[scale=.24, clip, trim=70 300 230 30]{img/bpeTimes/labelAle.pdf}
    			\end{minipage}
    			
    			(a)		
    	\end{minipage}
    	
       	\begin{minipage}{1\textwidth}
    			\centering
    			\begin{minipage}{0.8\textwidth}
    				\centering
    				\includegraphics[width=1\linewidth]{img/bpeTimes/aleatorio/marknewman-condmat.pdf}
    			\end{minipage}
    			\begin{minipage}{0.15\textwidth}
    				\centering
    				\includegraphics[scale=.24, clip, trim=70 300 230 30]{img/bpeTimes/labelAle.pdf}
    			\end{minipage}
    			
    			(b)		
    	\end{minipage}
    	
    \caption{BPE y tiempo de acceso aleatorio en microsegundos de cada algoritmo, para los grafos marknewman-astro y marknewman-condmat.}
    \label{fig:bpetAle1}
\end{figure}

\begin{figure}
    	\centering
    	\begin{minipage}{1\textwidth}
    			\centering
    			\begin{minipage}{0.8\textwidth}
    				\centering
    				\includegraphics[width=1\linewidth]{img/bpeTimes/aleatorio/dblp-2010.pdf}
    			\end{minipage}
    			\begin{minipage}{0.15\textwidth}
    				\centering
    				\includegraphics[scale=.24, clip, trim=70 200 300 40]{img/bpeTimes/labelAle.pdf}
    			\end{minipage}
    			
    			(a)		
    	\end{minipage}
    	
       	\begin{minipage}{1\textwidth}
    			\centering
    			\begin{minipage}{0.8\textwidth}
    				\centering
    				\includegraphics[width=1\linewidth]{img/bpeTimes/aleatorio/dblp-2011.pdf}
    			\end{minipage}
    			\begin{minipage}{0.15\textwidth}
    				\centering
    				\includegraphics[scale=.24, clip, trim=70 200 300 40]{img/bpeTimes/labelAle.pdf}
    			\end{minipage}
    			
    			(b)		
    	\end{minipage}
    	
    \caption{BPE y tiempo de acceso aleatorio en microsegundos de cada algoritmo, para los grafos dblp-2010 y dblp-2010.}
    \label{fig:bpetAle2}
\end{figure}

\begin{figure}
    	\centering
    	\begin{minipage}{1\textwidth}
    			\centering
    			\begin{minipage}{0.8\textwidth}
    				\centering
    				\includegraphics[width=1\linewidth]{img/bpeTimes/aleatorio/snap-dblp.pdf}
    			\end{minipage}
    			\begin{minipage}{0.15\textwidth}
    				\centering
    				\includegraphics[scale=.24, clip, trim=70 300 230 30]{img/bpeTimes/labelAle.pdf}
    			\end{minipage}
    			
    			(a)		
    	\end{minipage}
    	
       	\begin{minipage}{1\textwidth}
    			\centering
    			\begin{minipage}{0.8\textwidth}
    				\centering
    				\includegraphics[width=1\linewidth]{img/bpeTimes/aleatorio/snap-amazon.pdf}
    			\end{minipage}
    			\begin{minipage}{0.15\textwidth}
    				\centering
    				\includegraphics[scale=.24, clip, trim=70 300 230 30]{img/bpeTimes/labelAle.pdf}
    			\end{minipage}
    			
    			(b)		
    	\end{minipage}
    	
    \caption{BPE y tiempo de acceso aleatorio en microsegundos de cada algoritmo, para los grafos snap-dblp y snap-amazon.}
    \label{fig:bpetAle3}
\end{figure}

\begin{figure}
    	\centering
    	\begin{minipage}{1\textwidth}
    			\centering
    			\begin{minipage}{0.8\textwidth}
    				\centering
    				\includegraphics[width=1\linewidth]{img/bpeTimes/aleatorio/coPapersDBLP.pdf}
    			\end{minipage}
    			\begin{minipage}{0.15\textwidth}
    				\centering
    				\includegraphics[scale=.24, clip, trim=70 200 300 40]{img/bpeTimes/labelAle.pdf}
    			\end{minipage}
    			
    			(a)		
    	\end{minipage}
    	
       	\begin{minipage}{1\textwidth}
    			\centering
    			\begin{minipage}{0.8\textwidth}
    				\centering
    				\includegraphics[width=1\linewidth]{img/bpeTimes/aleatorio/coPapersCiteseer.pdf}
    			\end{minipage}
    			\begin{minipage}{0.15\textwidth}
    				\centering
    				\includegraphics[scale=.24, clip, trim=70 200 300 40]{img/bpeTimes/labelAle.pdf}
    			\end{minipage}
    			
    			(b)		
    	\end{minipage}
    	
    \caption{BPE y tiempo de acceso aleatorio en microsegundos de cada algoritmo, para los grafos coPapersDBLP y coPapersCiteseer.}
    \label{fig:bpetAle4}
\end{figure}


\begin{figure}
    	\centering
    	\begin{minipage}{1\textwidth}
    			\centering
    			\begin{minipage}{0.8\textwidth}
    				\centering
    				\includegraphics[width=1\linewidth]{img/bpeTimes/secuencial/marknewman-astro.pdf}
    			\end{minipage}
    			\begin{minipage}{0.15\textwidth}
    				\centering
    				\includegraphics[scale=.24, clip, trim=70 290 290 30]{img/bpeTimes/labelSec.pdf}
    			\end{minipage}
    			
    			(a)		
    	\end{minipage}
    	
       	\begin{minipage}{1\textwidth}
    			\centering
    			\begin{minipage}{0.8\textwidth}
    				\centering
    				\includegraphics[width=1\linewidth]{img/bpeTimes/secuencial/marknewman-condmat.pdf}
    			\end{minipage}
    			\begin{minipage}{0.15\textwidth}
    				\centering
    				\includegraphics[scale=.24, clip, trim=70 290 290 30]{img/bpeTimes/labelSec.pdf}
    			\end{minipage}
    			
    			(b)		
    	\end{minipage}
    	
    \caption{BPE y tiempo de reconstrucción secuencial en segundos de cada algoritmo, para los grafos marknewman-astro y marknewman-condmat.}
    \label{fig:bpetSec1}
\end{figure}

\begin{figure}
    	\centering
    	\begin{minipage}{1\textwidth}
    			\centering
    			\begin{minipage}{0.8\textwidth}
    				\centering
    				\includegraphics[width=1\linewidth]{img/bpeTimes/secuencial/dblp-2010.pdf}
    			\end{minipage}
    			\begin{minipage}{0.15\textwidth}
    				\centering
    				\includegraphics[scale=.24, clip, trim=70 290 290 30]{img/bpeTimes/labelSec.pdf}
    			\end{minipage}
    			
    			(a)		
    	\end{minipage}
    	
       	\begin{minipage}{1\textwidth}
    			\centering
    			\begin{minipage}{0.8\textwidth}
    				\centering
    				\includegraphics[width=1\linewidth]{img/bpeTimes/secuencial/dblp-2011.pdf}
    			\end{minipage}
    			\begin{minipage}{0.15\textwidth}
    				\centering
    				\includegraphics[scale=.24, clip, trim=70 290 290 30]{img/bpeTimes/labelSec.pdf}
    			\end{minipage}
    			
    			(b)		
    	\end{minipage}
    	
    \caption{BPE y tiempo de reconstrucción secuencial en segundos de cada algoritmo, para los grafos dblp-2010 y dblp-2011.}
    \label{fig:bpetSec2}
\end{figure}

\input{figs/bpetSec3}
\input{figs/bpetSec4}



%%%%%%%%%%%%%%%%%%%%%%%%%%%%%%%%%%%%%%%%%%%%%%%%%%%%%%%%%%%%%%%%%%%%%%%%%%%%%%%%
%
% Paso 17: Conclusiones
%
% 
%%%%%%%%%%%%%%%%%%%%%%%%%%%%%%%%%%%%%%%%%%%%%%%%%%%%%%%%%%%%%%%%%%%%%%%%%%%%%%%%


\chapter{CONCLUSIONES}\label{chap:conclusion}
\vskip 3.0ex

%En progreso...

Este trabajo se enfoca en desarrollar un método de compresión de grafos basado en clustering de cliques maximales, apuntado a grafos no dirigidos. Se logra llegar a una estructura compacta final que comprime un grafo y permite responder consultas sin tener que descomprimir para ello.

Entrando en detalle, el nivel de compresión logrado, medido en BPE, es mejor al estado del arte (ver Tabla~\ref{table:BPEcomp}), superando en todos los casos estudiados a los otros algoritmos. Incluso puntualmente para el caso de los grafos \texttt{coPapersDBLP} y \texttt{coPapersCiteseer}, los cuales poseen los coeficiente de clusterización (0,80 y 0,83) y transitividad (0,65 y 0,77) más altos (ver Tabla~\ref{table:gafros3}, se logran los BPE de 0,80 y 0,52 respectivamente, lo que es muy eficiente. Si bien este resultado en la compresión es muy positivo, es necesario reconocer el efecto que conlleva en los tiempos de acceso 

El tiempo de acceso aleatorio, medido usando el Algoritmo~\ref{alg:neighbors} recuperando vecinos para un millón de nodos, en varios casos se logran mejores resultados que k2tree con orden del grafo original, pero no usando BFS, donde solo para el grafo \texttt{dblp-2010} logra un tiempo similar, en los demás es mayor. Más lejos aún comparando con AD o Webgraph, donde en el mejor de los casos los tiempos logrados con respecto a AD son del orden del doble, para \texttt{marknewman-condmat}, \texttt{dblp-2010} y \texttt{coPapersDBLP} (ver Tabla~\ref{table:timesRandom}).

Para el tiempo de reconstrucción secuencial, medido usando el Algoritmo~\ref{alg:sequential}, solo para los grafos pequeños \texttt{marknewman-astro} y \texttt{marknewman-condmat} se obtienen resultados mejores con respecto a WebGraph, en ambos casos casi la mitad. Pero en general, tampoco se logra un tiempo mejor a los algoritmos en comparación, y para los grafos \texttt{coPapersDBLP} y \texttt{coPapersCiteseer} que presentan la mejor compresión, se obtienen tiempos mucho más altos (ver Tabla~\ref{table:timesSecuencial}).

Con esto en consideración, una buena aplicación para el método propuesto son dispositivos donde el espacio para guardar el grafo sea limitado, como dispositivos móviles con poca RAM y espacio en disco, donde se pueden almacenar los grafos usando una compresión muy eficiente, y que permitirá responder consultas sin ocupar mucho espacio extra y en un tiempo algo mayor. Esto no es menor, ya que sin esta opción de compresión, y pese a su costo en tiempos de acceso, no se podría almacenar los grafos en dichos dispositivos de otra manera.

Además, esta estructura compacta permite obtener el listado de cliques maximales directamente de ella, sin tener que descomprimir el grafo completo, y pese a que se necesita generar este listado antes de su construcción, una vez comprimido permite listar los cliques de manera rápida (ver Tabla~\ref{table:constructTimes}). Esto, junto con el nivel de compresión ya mencionado, sirve para trabajar con dispositivos de memoria acotada en variadas aplicaciones biológicas \cite{eblen2012maximum, hendrix2010theoretical}, entre otras \cite{bomze1999maximum}.

Como trabajo futuro, se puede explorar cómo mejorar los tiempos de acceso de esta estructura, por ejemplo encontrar nuevas funciones de ranking en la heurística de  clusterización para mejorar la relación entre compresión y tiempos de acceso. Otra opción es explotar el potencial de paralelismo que posee la estructura compacta, ya que cada partición se puede acceder de manera simultánea, y cada comparación de bytes dentro de las particiones se puede optimizar usando instrucciones paralelas, como SIMD\footnote{SIMD: Single Instruction, Multiple Data. Una Instrucción, múltiples datos.}.



%%%%%%%%%%%%%%%%%%%%%%%%%%%%%%%%%%%%%%%%%%%%%%%%%%%%%%%%%%%%%%%%%%%%%%%%%%%%%%%%
%
% Paso 18: Bibliografía
%
% Para compilar seguir esta secuencia: PDFLaTeX, BibTeX, PDFLaTeX, PDFLaTeX
%%%%%%%%%%%%%%%%%%%%%%%%%%%%%%%%%%%%%%%%%%%%%%%%%%%%%%%%%%%%%%%%%%%%%%%%%%%%%%%%
% Para la presentación de la bibliografía se recomienda seguir el estilo  propuesto por la American Psychological Association (APA), cuyas normas están  disponibles en la "GUÍA BREVE PARA LA PRESENTACIÓN DE REFERENCIAS Y CITAS  BIBLIOGRÁFICAS", descargable desde el sitio web de Sibudec.
%%%%%%%%%%%%%%%%%%%%%%%%%%%%%%%%%%%%%%%%%%%%%%%%%%%%%%%%%%%%%%%%%%%%%%%%%%%%%%%%
\bibliographystyle{plain}
\bibliography{biblio}


%%%%%%%%%%%%%%%%%%%%%%%%%%%%%%%%%%%%%%%%%%%%%%%%%%%%%%%%%%%%%%%%%%%%%%%%%%%%%%%%
%
% Paso 19: Glosario
%
% Opcional
%%%%%%%%%%%%%%%%%%%%%%%%%%%%%%%%%%%%%%%%%%%%%%%%%%%%%%%%%%%%%%%%%%%%%%%%%%%%%%%%


%%%%%%%%%%%%%%%%%%%%%%%%%%%%%%%%%%%%%%%%%%%%%%%%%%%%%%%%%%%%%%%%%%%%%%%%%%%%%%%%
%
% Paso 20: Anexos
%
% Opcional
%%%%%%%%%%%%%%%%%%%%%%%%%%%%%%%%%%%%%%%%%%%%%%%%%%%%%%%%%%%%%%%%%%%%%%%%%%%%%%%%
%\begin{appendices}
%\let\cleardoublepage\clearpage
%\chapter{Detalle de distribución del grado de los vértices para cada grafo.}
\label{anexo:grades}

\centering
\textbf{Anexo \thechapter:  1 de 4}
\begin{minipage}{1\textwidth}
    \centering
    \includegraphics[width=.9\linewidth]{img/grades/marknewman-astro.pdf} \\
    \includegraphics[width=.9\linewidth]{img/grades/marknewman-condmat.pdf} \\
\end{minipage}

\newpage
\centering
\textbf{Anexo \thechapter:  2 de 4}
\begin{minipage}{1\textwidth}
    \centering
    \includegraphics[width=.9\linewidth]{img/grades/dblp-2010.pdf} \\
    \includegraphics[width=.9\linewidth]{img/grades/dblp-2011.pdf} \\
\end{minipage}

\newpage
\centering
\textbf{Anexo \thechapter:  3 de 4}
\begin{minipage}{1\textwidth}
    \centering
    \includegraphics[width=.9\linewidth]{img/grades/snap-dblp.pdf} \\
    \includegraphics[width=.9\linewidth]{img/grades/snap-amazon.pdf} \\
\end{minipage}

\newpage
\centering
\textbf{Anexo \thechapter:  4 de 4}
\begin{minipage}{1\textwidth}
    \centering
    \includegraphics[width=.9\linewidth]{img/grades/coPapersDBLP.pdf} \\
    \includegraphics[width=.9\linewidth]{img/grades/coPapersCiteseer.pdf} \\
\end{minipage}

%\chapter{Detalle de CDF para bytes por vértice en estructuras compactas, para cada función de ranking}
\label{anexo:cdf}

\centering
\textbf{Anexo \thechapter:  1 de 4}
\begin{minipage}{1\textwidth}
    \centering
    \includegraphics[width=.9\linewidth]{img/cdf/marknewman-astro.pdf} \\
    \includegraphics[width=.9\linewidth]{img/cdf/marknewman-condmat.pdf} \\		
\end{minipage}

\centering
\begin{minipage}{1\textwidth}
    \centering
	\textbf{Anexo \thechapter:  2 de 4}
    \includegraphics[width=.9\linewidth]{img/cdf/dblp-2010.pdf} \\
    \includegraphics[width=.9\linewidth]{img/cdf/dblp-2011.pdf} \\
\end{minipage}

\centering
\begin{minipage}{1\textwidth}
    \centering
    \textbf{Anexo \thechapter:  3 de 4}
    \includegraphics[width=.9\linewidth]{img/cdf/snap-dblp.pdf} \\
    \includegraphics[width=.9\linewidth]{img/cdf/snap-amazon.pdf} \\
\end{minipage}

\centering
\begin{minipage}{1\textwidth}
    \centering
    \textbf{Anexo \thechapter:  4 de 4}
    \includegraphics[width=.9\linewidth]{img/cdf/coPapersDBLP.pdf} \\
    \includegraphics[width=.9\linewidth]{img/cdf/coPapersCiteseer.pdf} \\
\end{minipage}

%\end{appendices}



\end{document}