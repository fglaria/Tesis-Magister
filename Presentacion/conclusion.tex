\begin{frame}
\frametitle{Conclusiones}

Se desarrolla método de compresión:
\begin{itemize}
	\item Para grafos no dirigidos y poco densos.
	\item Basado en clustering de cliques maximales.
	\item Usando estructuras compactas.
\end{itemize}

Estructura comprimida permite responder consultas:
\begin{itemize}
	\item Reconstrucción del grafo original.
	\item Listado de vecinos de un nodo.
	\item Consultar vecindad de dos nodos.
	\item Generar listado de cliques maximales.
\end{itemize}

\end{frame}


\begin{frame}
\frametitle{Conclusiones (2)}

Nivel de compresión competitivo al estado del arte
\begin{itemize}
	\item Solo superado por k2tree.
\end{itemize}

Buen tiempo de acceso aleatorio
\begin{itemize}
	\item Competitivo con k2tree.
	\item Otros algoritmos logran mejor tiempo.
\end{itemize}

Menor desempeño en reconstrucción secuencial.
\begin{itemize}
	\item Algunos casos compiten con Webgraph.
	\item Otros algoritmos logran mejor tiempo.
\end{itemize}


\end{frame}

\begin{frame}
\frametitle{Conclusiones (3)}

Posible aplicación:
\begin{itemize}
	\item Dispositivos con poca memoria.
	\item Responder consultas sin descomprimir.
\end{itemize}

Trabajo futuro:
\begin{itemize}
	\item Mejorar tiempos de acceso.
		\begin{itemize}
			\footnotesize
			\item Nuevas heurísticas en funciones de ranking.
		\end{itemize}
	\item Potencial de paralelismo.
		\begin{itemize}
			\footnotesize
			\item Acceso paralelo por partición.
			\item Comparar bytes usando instrucciones paralelas (SIMD\footnote{SIMD: Single Instruction, Multiple Data. Una instrucción, múltiples datos.}).
		\end{itemize}
\end{itemize}


\end{frame}