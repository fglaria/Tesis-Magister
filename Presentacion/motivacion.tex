\begin{frame}{•}
\frametitle{Motivación}

Gran crecimiento en grafos de redes sociales y de la web.
\begin{itemize}
	\item Estimación sitios indexados: 5,68 mil millones\footnote{\url{http://www.worldwidewebsize.com/}, consultado el 07 de agosto del 2019.}.
	\item Usuarios activos diarios en Facebook: 1,56 mil millones. Crecimiento de 8\% anual\footnote{\href{https://investor.fb.com/investor-news/press-release-details/2019/Facebook-Reports-First-Quarter-2019-Results/default.aspx}{\texttt{https://investor.fb.com}}, informe de resultados del primer trimestre del 2019.}.
\end{itemize}

Estos grafos son muy usados por algoritmos de ranking, detección de SPAM, detección de comunidades y actores relevantes, entre otros.

Alto costo en recursos que demanda su procesamiento.
\begin{itemize}
	\item Principalmente espacio en memoria.
	\item Jerarquía de memoria penaliza tiempo de acceso a datos alejados de unidades de procesamiento.
\end{itemize}

\end{frame}

\begin{frame}{•}
\frametitle{Motivación (2)}

Proponer estructuras compactas que permitan navegación basado en consultas básicas.
\\

Modelo propuesto enumera cliques maximales de un grafo, y luego los representa en una estructura compacta.
\end{frame}
