\begin{frame}
\frametitle{Etapa 2: Particionar listado de cliques}

\begin{problem}
	\label{def:findPartitions}
	Encontrar particiones de cliques para el grafo de cliques $CG_{\mathcal{C}}$.
	
	Dado un grafo de cliques $CG_{\mathcal{C}} = (V_{\mathcal{C}}, E_{\mathcal{C}})$, encontrar un set de particiones de cliques $\mathcal{C}\mathcal{P} = \{cp_{1}, cp_{2}, ..., cp_{M}\}$ de $CG_{\mathcal{C}}(V_{\mathcal{C}}, E_{\mathcal{C}})$ con $M \geq 1$, tal que
	\begin{enumerate}
		%\item $\bigcup_{i \in \mathcal{C}\mathcal{P}} cp_{i} = CG_{i}$ \label{item:particiones1}
		\item $\bigcup\limits_{i = 1}^{M} cp_{i} = CG_{i}$ \label{item:particiones1}
		\item $cp_{i} \cap cp_{j} = \varnothing$ para $i \neq j$ \label{item:particiones2}
		\item cualquier $cp_{i} \in \mathcal{C}\mathcal{P}$ es un subgrafo de $CG_{\mathcal{C}}(V_{\mathcal{C}}, E_{\mathcal{C}})$ inducido por el subset de vértices en $cp_{i}$ \label{item:particiones3}
	\end{enumerate}
	
\end{problem}

\end{frame}


\begin{frame}
\frametitle{Etapa 2: Particionar listado de cliques (2)}

%\begin{itemize}
%	\item Definir una heurística que permita agruparlos en particiones eficientes.
%	\item Que exploten dicha redundancia de vértices en los cliques maximales.
%\end{itemize}

Definir una heurística que permita agruparlos en particiones eficientes, que exploten dicha redundancia de vértices en los cliques maximales.

\begin{definition} 
	\label{def:rankingFunctions}
	Función de ranking
	
	Dado un grafo $G = (V, E)$ y $\mathcal{C} = \{c_{1}, c_{2}, ..., c_{N} \}$ el conjunto de tamaño $N$ de cliques maximales que cubren $G$, una función de ranking es una función $r: V \rightarrow \mathbb{R}^{+}$ que retorna un valor de puntuación para cada vértice $v \in V$.
\end{definition}

\end{frame}
